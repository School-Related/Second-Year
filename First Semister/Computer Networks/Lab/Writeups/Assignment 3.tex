% This is a Basic Assignment Paper but with like Code and stuff allowed in it. 

\documentclass[11pt]{article}

% Preamble

\usepackage[margin=1in]{geometry}
\usepackage{amsfonts, amsmath, amssymb}
\usepackage{fancyhdr, float, graphicx}
\usepackage[utf8]{inputenc} % Required for inputting international characters
\usepackage[T1]{fontenc} 
\usepackage{fouriernc} % Use the New Century Schoolbook font
\usepackage[nottoc, notlot, notlof]{tocbibind}
\usepackage{listings}
\usepackage{xcolor}
\usepackage{float}

\definecolor{codegreen}{rgb}{0,0.6,0}
\definecolor{codegray}{rgb}{0.5,0.5,0.5}
\definecolor{codepurple}{rgb}{0.58,0,0.82}
\definecolor{backcolour}{rgb}{0.95,0.95,0.92}

\lstdefinestyle{mystyle}{
    backgroundcolor=\color{backcolour},   
    commentstyle=\color{codegreen},
    keywordstyle=\color{magenta},
    numberstyle=\tiny\color{codegray},
    stringstyle=\color{codepurple},
    basicstyle=\ttfamily\footnotesize,
    breakatwhitespace=false,         
    breaklines=true,                 
    captionpos=b,                    
    keepspaces=true,                 
    numbers=left,                    
    numbersep=5pt,                  
    showspaces=false,                
    showstringspaces=false,
    showtabs=false,                  
    tabsize=2
}

\lstset{style=mystyle}

% Header and Footer     Reset functions

\pagestyle{fancy}
\fancyhead{}
\fancyfoot{}
\fancyhead[L]{\textit{\Large{Computer Networks Assignment 3}}}
%\fancyhead[R]{\textit{something}}
\fancyfoot[C]{\thepage}
\renewcommand{\footrulewidth}{1pt}



% Other Doc Editing
% \parindent 0ex
%\renewcommand{\baselinestretch}{1.5}

\begin{document}

\begin{titlepage}
	\centering

	%---------------------------NAMES-------------------------------

	\huge\textsc{
		MIT World Peace University
	}\\

	\vspace{0.75\baselineskip} % space after Uni Name

	\LARGE{
		Computer Networks\\
		Second Year B. Tech, Semester 3
	}

	\vfill % space after Sub Name

	%--------------------------TITLE-------------------------------

	\rule{\textwidth}{1.6pt}\vspace*{-\baselineskip}\vspace*{2pt}
	\rule{\textwidth}{0.6pt}
	\vspace{0.75\baselineskip} % Whitespace above the title



	\huge{\textsc{
			Configuration of a Virtual LAN
		}} \\



	\vspace{0.5\baselineskip} % Whitespace below the title
	\rule{\textwidth}{0.6pt}\vspace*{-\baselineskip}\vspace*{2.8pt}
	\rule{\textwidth}{1.6pt}

	\vspace{1\baselineskip} % Whitespace after the title block

	%--------------------------SUBTITLE --------------------------	

	\LARGE\textsc{
		Practical Report\\
		Assignment 3
	} % Subtitle or further description
	\vfill

	%--------------------------AUTHOR-------------------------------

	Prepared By
	\vspace{0.5\baselineskip} % Whitespace before the editors

	\Large{
		Krishnaraj Thadesar \\
		Cyber Security and Forensics\\
		Batch A1, PA 20
	}


	\vspace{0.5\baselineskip} % Whitespace below the editor list
	\today

\end{titlepage}


\tableofcontents
\thispagestyle{empty}
\clearpage


\setcounter{page}{1}

\section{Aim and Objectives}
To Design and configure a virtual LAN using Packet Tracer and To understand the concept of VLAN and implement it using packet tracer.

\section{Devices}

\subsection{Devices Used}
\begin{enumerate}
	\item 1 Generic Switch
	\item 2 Switch 2960 with 24 LAN Ports
	\item 6 Generic PCs
	\item 4 Laptops
\end{enumerate}

\subsection{Device Info and IP Addresses}
\begin{table}[H]
	\begin{tabular}{|l|l|l|c|}
		\hline
		\textbf{No} & \textbf{Device Name} & \textbf{Model} & \textbf{IP}       \\ \hline
		\textbf{1}  & Switch0              & 2950-24        & - \\ \hline
		2           & PC0 - CS             & PC-PT          & 192.168.0.1 \\ \hline
		3           & PC1 - ECE            & PC-PT          & 192.168.0.5 \\ \hline
		4           & PC2 MECH             & PC-PT          & 192.168.0.2 \\ \hline
		5           & Switch1              & 2950-24        & - \\ \hline
		6           & PC3 CS               & PC-PT          & 192.168.0.3 \\ \hline
		7           & PC4 PH               & PC-PT          & 192.168.0.4 \\ \hline
		8           & Switch2              & 2950-24        & - \\ \hline
		9           & Laptop0 CS           & Laptop-PT      & 192.168.0.6 \\ \hline
		10          & Laptop1 PH           & Laptop-PT      & 192.168.0.9 \\ \hline
		11          & Laptop2 MECH         & Laptop-PT      & 192.168.0.8 \\ \hline
		12          & Laptop3 ECE          & Laptop-PT      & 192.168.0.7 \\ \hline
		13          & PC5 ECE              & PC-PT          & 192.168.0.10 \\ \hline
	\end{tabular}
\end{table}


\section{Cables}
\begin{enumerate}
	\item Straight LAN Cable to connect unlike Devices
	\item Crossover LAN Cable to connect like Devices
\end{enumerate}

\section{Procedure to Configure VLAN}
\begin{enumerate}
	\item Create a Simple network with a switch and a few PCs
	\item Create as many other networks you want, connect a few PCS or laptops to it. Use a generic Switch.
	\item Name the PCs and Laptops according to some virtual division you want to make, be it different divisions of a single institution, wings in a hospital or anything. 
	\item Connect the switches to each toher using a crossover cable, and the PCs to the switch using a Straight cable. 
	\item Click on the Switch and open its terminal, or GUI, where you can add the VLAN name and Number. Add the VLANS respective to the ones you have in your network, to all the switches either via its GUI or terminal using commands given below. 
	\item Select Each interface and set its particular VLAN. You can do this in the terminal for each switch or with the GUI. 
\end{enumerate}

\section{Commands}
\begin{verbatim}
	# enable
	# configure terminal
	# exitverbatim
	# vlan 20 name Mechanical
	# vlan 10 name CS
	# vlan 30 name Pharma
	# terminal Show VLAN
	# vlan database
	# interface F0/2
	# switchport access vlan <VLAN_NO>
	# exit
	# interface F0/1 // the one connected to another switch
	# switchport mode trunk
\end{verbatim}

\section{Output}
\subsection{Switch 1}
\begin{lstlisting}
Switch#show vlan

VLAN Name                             Status    Ports
---- -------------------------------- --------- -------------------------------
1    default                          active    Fa0/1, Fa0/7, Fa0/8, Fa0/9
                                                Fa0/11, Fa0/12, Fa0/13, Fa0/14
                                                Fa0/15, Fa0/16, Fa0/17, Fa0/18
                                                Fa0/19, Fa0/20, Fa0/21, Fa0/22
                                                Fa0/23, Fa0/24
10   CS                               active    Fa0/6, Fa0/10
11   ECE                              active    Fa0/3
12   MECH                             active    Fa0/4
13   PH                               active    Fa0/2
1002 fddi-default                     active    
1003 token-ring-default               active    
1004 fddinet-default                  active    
1005 trnet-default                    active    

VLAN Type  SAID       MTU   Parent RingNo BridgeNo Stp  BrdgMode Trans1 Trans2
---- ----- ---------- ----- ------ ------ -------- ---- -------- ------ ------
1    enet  100001     1500  -      -      -        -    -        0      0
10   enet  100010     1500  -      -      -        -    -        0      0
11   enet  100011     1500  -      -      -        -    -        0      0
12   enet  100012     1500  -      -      -        -    -        0      0
13   enet  100013     1500  -      -      -        -    -        0      0
1002 fddi  101002     1500  -      -      -        -    -        0      0   
1003 tr    101003     1500  -      -      -        -    -        0      0   
1004 fdnet 101004     1500  -      -      -        ieee -        0      0   
1005 trnet 101005     1500  -      -      -        ibm  -        0      0   

VLAN Type  SAID       MTU   Parent RingNo BridgeNo Stp  BrdgMode Trans1 Trans2
---- ----- ---------- ----- ------ ------ -------- ---- -------- ------ ------

Remote SPAN VLANs
------------------------------------------------------------------------------

Primary Secondary Type              Ports
------- --------- ----------------- ------------------------------------------
\end{lstlisting}

\subsection{Switch 2}

\begin{lstlisting}
Switch#show vlan

VLAN Name                             Status    Ports
---- -------------------------------- --------- -------------------------------
1    default                          active    Fa0/7, Fa0/8, Fa0/9, Fa0/10
                                                Fa0/11, Fa0/12, Fa0/13, Fa0/14
                                                Fa0/15, Fa0/16, Fa0/17, Fa0/18
                                                Fa0/19, Fa0/20, Fa0/21, Fa0/22
                                                Fa0/23, Fa0/24
10   CS                               active    Fa0/5
11   ECE                              active    Fa0/6
12   MECH                             active    Fa0/4
13   PH                               active    Fa0/3
1002 fddi-default                     active    
1003 token-ring-default               active    
1004 fddinet-default                  active    
1005 trnet-default                    active    

VLAN Type  SAID       MTU   Parent RingNo BridgeNo Stp  BrdgMode Trans1 Trans2
---- ----- ---------- ----- ------ ------ -------- ---- -------- ------ ------
1    enet  100001     1500  -      -      -        -    -        0      0
10   enet  100010     1500  -      -      -        -    -        0      0
11   enet  100011     1500  -      -      -        -    -        0      0
12   enet  100012     1500  -      -      -        -    -        0      0
13   enet  100013     1500  -      -      -        -    -        0      0
1002 fddi  101002     1500  -      -      -        -    -        0      0   
1003 tr    101003     1500  -      -      -        -    -        0      0   
1004 fdnet 101004     1500  -      -      -        ieee -        0      0   
1005 trnet 101005     1500  -      -      -        ibm  -        0      0   

VLAN Type  SAID       MTU   Parent RingNo BridgeNo Stp  BrdgMode Trans1 Trans2
---- ----- ---------- ----- ------ ------ -------- ---- -------- ------ ------

Remote SPAN VLANs
------------------------------------------------------------------------------

Primary Secondary Type              Ports
------- --------- ----------------- ------------------------------------------
\end{lstlisting}


\subsection{Switch 3}
\begin{lstlisting}
Switch#show vlan
	
VLAN Name                             Status    Ports
---- -------------------------------- --------- -------------------------------
1    default                          active    Fa0/8, Fa0/9, Fa0/10, Fa0/11
                                                Fa0/12, Fa0/13, Fa0/14, Fa0/15
                                                Fa0/16, Fa0/17, Fa0/18, Fa0/19
                                                Fa0/20, Fa0/21, Fa0/22, Fa0/23
                                                Fa0/24
10   CS                               active    Fa0/3, Fa0/6
11   ECE                              active    Fa0/7
12   MECH                             active    Fa0/4
13   PH                               active    Fa0/2, Fa0/5
1002 fddi-default                     active    
1003 token-ring-default               active    
1004 fddinet-default                  active    
1005 trnet-default                    active    

VLAN Type  SAID       MTU   Parent RingNo BridgeNo Stp  BrdgMode Trans1 Trans2
---- ----- ---------- ----- ------ ------ -------- ---- -------- ------ ------
1    enet  100001     1500  -      -      -        -    -        0      0
10   enet  100010     1500  -      -      -        -    -        0      0
11   enet  100011     1500  -      -      -        -    -        0      0
12   enet  100012     1500  -      -      -        -    -        0      0
13   enet  100013     1500  -      -      -        -    -        0      0
1002 fddi  101002     1500  -      -      -        -    -        0      0   
1003 tr    101003     1500  -      -      -        -    -        0      0   
1004 fdnet 101004     1500  -      -      -        ieee -        0      0   
1005 trnet 101005     1500  -      -      -        ibm  -        0      0   

VLAN Type  SAID       MTU   Parent RingNo BridgeNo Stp  BrdgMode Trans1 Trans2
---- ----- ---------- ----- ------ ------ -------- ---- -------- ------ ------

Remote SPAN VLANs
------------------------------------------------------------------------------

Primary Secondary Type              Ports
------- --------- ----------------- ------------------------------------------
\end{lstlisting}
\section{Platform}
\textbf{Operating System}: Arch Linux x86-64\\
\textbf{IDEs or Text Editors Used}: Visual Studio Code\\
\textbf{Programs Used}: Cisco Packet Tracer v8.2

\section{Connection Screenshot}


\begin{figure}[H]
	\centering
	\includegraphics[scale=0.35]{../Screenshots/Assignment_3_screenshot.png}
\end{figure}


\section{Conclusion}
A Virtual Local Area Network was implemented successfulyl with 3 switches and 4 Virtual LANS. The Concept and its uses were understood.
\end{document}