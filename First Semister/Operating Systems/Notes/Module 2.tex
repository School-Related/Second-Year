% This is a basic Math Paper

\documentclass[11pt]{article}
 
% Preamble

\usepackage[margin=1in]{geometry}
\usepackage{amsfonts, amsmath, amssymb}
\usepackage{fancyhdr, float, graphicx}
\usepackage[utf8]{inputenc} % Required for inputting international characters
\usepackage[T1]{fontenc} % Output font encoding for international characters
\usepackage{fouriernc} % Use the New Century Schoolbook font
\usepackage[nottoc, notlot, notlof]{tocbibind}
\usepackage{url}

% Header and Footer
\pagestyle{fancy}
\fancyhead{}
\fancyfoot{}
\fancyhead[L]{\textit{\Large{Operating Systems}}}
%\fancyhead[R]{\textit{something}}
\fancyfoot[C]{\thepage}
\renewcommand{\footrulewidth}{1pt}



% Other Doc Editing
% \parindent 0ex
%\renewcommand{\baselinestretch}{1.5}

\begin{document}

\begin{titlepage}
	\centering

	%---------------------------NAMES-------------------------------

	\huge\textsc{
		MIT World Peace University
	}\\

	\vspace{0.75\baselineskip} % space after Uni Name

	\LARGE{
		Computer Networks\\
		Second Year B.Tech Semister 3\\
		Academic Year 2022-23
	}

	\vfill % space after Sub Name

	%--------------------------TITLE-------------------------------

	\rule{\textwidth}{1.6pt}\vspace*{-\baselineskip}\vspace*{2pt}
	\rule{\textwidth}{0.6pt}
	\vspace{0.75\baselineskip} % Whitespace above the title



	\huge{\textsc{
			Operating Systems
		}} \\



	\vspace{0.5\baselineskip} % Whitespace below the title
	\rule{\textwidth}{0.6pt}\vspace*{-\baselineskip}\vspace*{2.8pt}
	\rule{\textwidth}{1.6pt}

	\vspace{1\baselineskip} % Whitespace after the title block

	%--------------------------SUBTITLE --------------------------	

	\LARGE\textsc{
		Notes from Tananbaum and Classes
	} % Subtitle or further description
	\vfill

	%--------------------------AUTHOR-------------------------------

	Prepared By
	\vspace{0.5\baselineskip} % Whitespace before the editors

	\Large{
		P34. Krishnaraj Thadesar\\
		\vspace{1cm}
		Batch A2
	}


	\vspace{0.5\baselineskip} % Whitespace below the editor list
	\today

\end{titlepage}

\clearpage
\tableofcontents
\clearpage


\section{Processes}

A process is an instance of a program in exeuction. It is an entity that can be assigned to and executed on a procesor.

\begin{enumerate}
	\item Process is compromised of Program Code
	\item Data
	\item Stack
	\item A number of attribute describing the state of process.
\end{enumerate}

\subsection{Process states}
\begin{enumerate}
	\item New
	\item Ready
	\item Running
	\item Waiting
	\item Terminated
	\item Suspended
\end{enumerate}


\subsubsection{Suspended State}
\begin{itemize}
	\item Process is faster than IO so man processes could be waiting for IO
	\item Swap this proces to disk (SSD/ HDD) to free up RAM memory.
	\item Ready or waiting state becomes suspended state when swapped to disk.
\end{itemize}

\section{Process Control Block}
It is a data structure maintained by the Operating System. It holds all necessary information related to Process.
Information Associated with each process is as follows:
\begin{enumerate}
	\item Process state
	\item Program Counter
	\item CPU Registers
	\item CPU Scheduling information
	\item Memory Management Information
	\item Accounting Information
	\item IO Status information
\end{enumerate}

\section{Switches}
\subsection{Context Switches}
\begin{enumerate}
	\item It switches the execution of a process to another, so for that it has to do some stuff,
	\item It saves the state of the first program, and then reloads the state of the next one.
	\item And only then it runs the next process. This takes time, and is a major disadvantage.
	\item It is a mode switch, but a mode switch isnt a context switch.
	\item It is a mode switch coz it requires you to switch mode from user to kernel.
\end{enumerate}

\section{Process Execution}
Consider three processes being executed, all are in the meory, plus the dispatcher.
\paragraph{Dispatcher}
Dispatcher is a small program which switches from one program to another. -

\section{Process Creation}
When a new process is created, the following happens:
\begin{enumerate}
	\item Allocates space to the process in memory
	\item Assign a unique Process ID to the Process
	\item A process control Block PCB gets associated with the process
	\item OS Maintains pointers to each process's PCB in a process table sothat it can access the PCB quickly.
\end{enumerate}

Reasons to create a Process
\begin{enumerate}
	\item New User Job
	\item Created by OS to provide a service
	\item Spawned by existing Process: The action of creating a new process at the explicit request of another process is called process spawning.
\end{enumerate}

After Creation
\begin{enumerate}
	\item Stay in the parent Process
	\item Transfer Control to the child process. The system call for that is called Fork. This child process inherits everything from the parent.
	\item Transfer control to another process.
\end{enumerate}


\subsection{fork()}
A system call fork() is used to create processes. It takes no arguments and returns a process ID.
The syntax for the fork system call
pid = fork();\\
in the Parent process, pid is the child process\\
In the child process, pid is 0

\begin{enumerate}
	\item It allocates a slot in the process table for the new process
	\item It assigns a unique ID number to the child process
	\item It makes a copy of the context of the parent process.
	\item It returns the ID number of the child to the parent process, and a 0 value to the child process process is assigned.
	\item It doesnt take any arguements
	\item Purpose of fork is to create a new process, which becomes the child process of the caller.
	\item After a process is created, both processes will execute the next instruction following the fork system call.
	\item To distinguist the parent from the child, the returned value of fork can be used.
	\item \begin{itemize}
		      \item fork() returns a negative value to the parent if the creation of the child process wasnt successful
		      \item 0 to the child process if successful, and the PID of thus generated child process to the parent process.
	      \end{itemize}
	\item Returned process id is of type PID defined in sys/types.h
	\item Process can se function getpid() to retrieve the process ID assigned to this process.
	\item Linux would make an exact copy of the parent's address space and give it to the child. Therefore the parent and child process will always have a separate address space.

\end{enumerate}

The OS will make two identical copies of address spaces for parent and child processes. So the parent and child processes have different address spaces.
A local variable is:
\begin{enumerate}
	\item Declared inside the process
	\item  Created when the process starts
	\item  Lost when the process terminates
\end{enumerate}

A global variable is:
\begin{enumerate}
	\item Declared outside the process
	\item Created as the process starts
	\item Lost when the program ends
\end{enumerate}

The process ID, i.e., PID of the child process created, is returned to the parent process. (In case of failure, -1 is returned to the parent process.) \\
Zero is returned to the child process. (If it fails, the child process is not created.) If a child process exits at that instant or is interrupted, a signal SIGCHLD is sent to the parent process. \\
Both parent and child processes independently execute the subsequent commands after the fork() system call.


\section{Process Termination}
\begin{itemize}
	\item All the resources held by process are released.
	\item All the information help in all data structures is removed.
	\item A process goes back to becoming a program and is stored on the secondary memory.
\end{itemize}


\section{Threads}
\begin{itemize}
	\item A thread is a part of a program.
	\item It is an execution unit within a process
	\item All threads of the same process share the same address space.
	\item All threas have separate stacks and individual Thread IDs.
	\item Thread is a lightweight process because: The contexxt switching between threads is inexpensive in terms of memory and resources.
	\item Even if 2 processes are communicating within each other, its is their threads that are communicating.
\end{itemize}

\section{Differences between thread and processes}
\begin{enumerate}
	\item Process is program in execution, thread is process in execution
	\item Inter process communication is slower than inter thread communication.
	\item They both have unique ids, unique pid, and unique thread id.
	\item Context switching is expensive in processes, its inexpensive in thread.
	\item Every process has its own memory address, but threads use the memory of the process that they belong to.
\end{enumerate}
\end{document}do