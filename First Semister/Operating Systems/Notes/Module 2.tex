% This is a basic Math Paper

\documentclass[11pt]{article}
 
% Preamble

\usepackage[margin=1in]{geometry}
\usepackage{amsfonts, amsmath, amssymb}
\usepackage{fancyhdr, float, graphicx}
\usepackage[utf8]{inputenc} % Required for inputting international characters
\usepackage[T1]{fontenc} % Output font encoding for international characters
\usepackage{fouriernc} % Use the New Century Schoolbook font
\usepackage[nottoc, notlot, notlof]{tocbibind}
\usepackage{url}

% Header and Footer
\pagestyle{fancy}
\fancyhead{}
\fancyfoot{}
\fancyhead[L]{\textit{\Large{Operating Systems}}}
%\fancyhead[R]{\textit{something}}
\fancyfoot[C]{\thepage}
\renewcommand{\footrulewidth}{1pt}



% Other Doc Editing
% \parindent 0ex
%\renewcommand{\baselinestretch}{1.5}

\begin{document}

\begin{titlepage}
	\centering

	%---------------------------NAMES-------------------------------

	\huge\textsc{
		MIT World Peace University
	}\\

	\vspace{0.75\baselineskip} % space after Uni Name

	\LARGE{
		Computer Networks\\
		Second Year B.Tech Semister 3\\
		Academic Year 2022-23
	}

	\vfill % space after Sub Name

	%--------------------------TITLE-------------------------------

	\rule{\textwidth}{1.6pt}\vspace*{-\baselineskip}\vspace*{2pt}
	\rule{\textwidth}{0.6pt}
	\vspace{0.75\baselineskip} % Whitespace above the title



	\huge{\textsc{
			Operating Systems
		}} \\



	\vspace{0.5\baselineskip} % Whitespace below the title
	\rule{\textwidth}{0.6pt}\vspace*{-\baselineskip}\vspace*{2.8pt}
	\rule{\textwidth}{1.6pt}

	\vspace{1\baselineskip} % Whitespace after the title block

	%--------------------------SUBTITLE --------------------------	

	\LARGE\textsc{
		Notes from Tananbaum and Classes
	} % Subtitle or further description
	\vfill

	%--------------------------AUTHOR-------------------------------

	Prepared By
	\vspace{0.5\baselineskip} % Whitespace before the editors

	\Large{
		P34. Krishnaraj Thadesar\\
		\vspace{1cm}
		Batch A2
	}


	\vspace{0.5\baselineskip} % Whitespace below the editor list
	\today

\end{titlepage}

\clearpage
\tableofcontents
\clearpage


\section{Processes}

A process is an instance of a program in exeuction. It is an entity that can be assigned to and executed on a procesor. 

\begin{enumerate}
    \item Process is compromised of Program Code
    \item Data
    \item Stack
    \item A number of attribute describing the state of process. 
\end{enumerate}

\subsection{Process states}
\begin{enumerate}
    \item New
    \item Ready
    \item Running
    \item Waiting
    \item Terminated
    \item Suspended
\end{enumerate}


\subsubsection{Suspended State}
\begin{itemize}
    \item Process is faster than IO so man processes could be waiting for IO
    \item Swap this proces to disk (SSD/ HDD) to free up RAM memory. 
    \item Ready or waiting state becomes suspended state when swapped to disk. 
\end{itemize}

\section{Process Control Block}
It is a data structure maintained by the Operating System. It holds all necessary information related to Process.
Information Associated with each process is as follows:
\begin{enumerate}
    \item Process state
    \item Program Counter
    \item CPU Registers
    \item CPU Scheduling information
    \item Memory Management Information
    \item Accounting Information
    \item IO Status information
\end{enumerate}

\section{Switches}
\subsection{Context Switches}
\begin{enumerate}
    \item It switches the execution of a process to another, so for that it has to do some stuff, 
    \item It saves the state of the first program, and then reloads the state of the next one. 
    \item And only then it runs the next process. This takes time, and is a major disadvantage. 
    \item It is a mode switch, but a mode switch isnt a context switch. 
    \item It is a mode switch coz it requires you to switch mode from user to kernel. 
\end{enumerate}

\section{Process Execution}
Consider three processes being executed, all are in the meory, plus the dispatcher. 
\paragraph{Dispatcher}
Dispatcher is a small program which switches from one program to another. -

\section{Process Creation}
When a new process is created, the following happens:
\begin{enumerate}
    \item Allocates space to the process in memory
    \item Assign a unique Process ID to the Process
    \item A process control Block PCB gets associated with the process
    \item OS Maintains pointers to each process's PCB in a process table sothat it can access the PCB quickly. 
\end{enumerate}

Reasons to create a Process
\begin{enumerate}
    \item New User Job
    \item Created by OS to provide a service
    \item Spawned by existing Process: The action of creating a new process at the explicit request of another process is called process spawning.
\end{enumerate}

After Creation
\begin{enumerate}
    \item Stay in the parent Process
    \item Transfer Control to the child process. The system call for that is called Fork. This child process inherits everything from the parent. 
    \item Transfer control to another process. 
\end{enumerate}


\subsection{fork()}
A system call fork() is used to create processes. It takes no arguments and returns a process ID. 
The syntax for the fork system call
pid = fork();\\
in the Parent process, pid is the child process\\
In the child process, pid is 0

\begin{enumerate}
    \item It allocates a slot in the process table for the new process
    \item It assigns a unique ID number to the child process
    \item It makes a copy of the context of the parent process. 
    \item It returns the ID number of the child to the parent process, and a 0 value to the child process process is assigned. 
\end{enumerate}

\end{document}