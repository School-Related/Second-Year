% This is a basic Math Paper

\documentclass[11pt]{article}

% Preamble

\usepackage[margin=1in]{geometry}
\usepackage{amsfonts, amsmath, amssymb}
\usepackage{fancyhdr, float, graphicx}
\usepackage[utf8]{inputenc} % Required for inputting international characters
\usepackage[T1]{fontenc} % Output font encoding for international characters
\usepackage{fouriernc} % Use the New Century Schoolbook font
\usepackage[nottoc, notlot, notlof]{tocbibind}
\usepackage{url}

% Header and Footer
\pagestyle{fancy}
\fancyhead{}
\fancyfoot{}
\fancyhead[L]{\textit{\Large{Object Oriented Programming with C++ and Java}}}
%\fancyhead[R]{\textit{something}}
\fancyfoot[C]{\thepage}
\renewcommand{\footrulewidth}{1pt}

\usepackage{listings}
\usepackage{xcolor}

\definecolor{codegreen}{rgb}{0,0.6,0}
\definecolor{codegray}{rgb}{0.5,0.5,0.5}
\definecolor{codepurple}{rgb}{0.58,0,0.82}
\definecolor{backcolour}{rgb}{0.95,0.95,0.92}

\lstdefinestyle{mystyle}{
    backgroundcolor=\color{backcolour},   
    commentstyle=\color{codegreen},
    keywordstyle=\color{magenta},
    numberstyle=\tiny\color{codegray},
    stringstyle=\color{codepurple},
    basicstyle=\ttfamily\footnotesize,
    breakatwhitespace=false,         
    breaklines=true,                 
    captionpos=b,                    
    keepspaces=true,                 
    numbers=left,                    
    numbersep=5pt,                  
    showspaces=false,                
    showstringspaces=false,
    showtabs=false,                  
    tabsize=2
}

\lstset{style=mystyle}



% Other Doc Editing
% \parindent 0ex
%\renewcommand{\baselinestretch}{1.5}

\begin{document}

\begin{titlepage}
	\centering

	%---------------------------NAMES-------------------------------

	\huge\textsc{
		MIT World Peace University
	}\\

	\vspace{0.75\baselineskip} % space after Uni Name

	\LARGE{
		Computer Networks\\
		Second Year B.Tech Semister 3\\
		Academic Year 2022-23
	}

	\vfill % space after Sub Name

	%--------------------------TITLE-------------------------------

	\rule{\textwidth}{1.6pt}\vspace*{-\baselineskip}\vspace*{2pt}
	\rule{\textwidth}{0.6pt}
	\vspace{0.75\baselineskip} % Whitespace above the title



	\huge{\textsc{
			Module 3\\ Exception Handling, File and IO Streams
		}} \\



	\vspace{0.5\baselineskip} % Whitespace below the title
	\rule{\textwidth}{0.6pt}\vspace*{-\baselineskip}\vspace*{2.8pt}
	\rule{\textwidth}{1.6pt}

	\vspace{1\baselineskip} % Whitespace after the title block

	%--------------------------SUBTITLE --------------------------	

	\LARGE\textsc{
		Notes
	} % Subtitle or further description
	\vfill

	%--------------------------AUTHOR-------------------------------

	Prepared By
	\vspace{0.5\baselineskip} % Whitespace before the editors

	\Large{
		P34. Krishnaraj Thadesar\\
		\vspace{1cm}
		Batch A2
	}


	\vspace{0.5\baselineskip} % Whitespace below the editor list
	\today

\end{titlepage}

\clearpage
\tableofcontents
\clearpage


\section{Exceptions}
\begin{itemize}
	\item An exception is an unusual often unpredictable event, detectable by software or hardware, that requires special processing occuring at runtime. 
	\item In C++, a variable or a class object represents an exceptional \textbf{event.}
	\item \textbf{Exceptions}, Indicate problems that occur during a programs execution. 
	\item They Occur infrequently. 
	\item If we dont handle them, the program crashes or falls into unkonwn state.
	\item To fix this we need an exception handler, which is a secction of program code that is designed to execute when a particular exception occurs. 
	\item This can resolve exceptions, alow a program to continue executing or notify the user of the program. It could even help you terminal teh program in a controlled manner. 
	\item It makes your code \textit{clear, robust and fault-tolerant}
\end{itemize}

\subsection{Standard Exceptions}
\begin{itemize}
	\item Excetions thrown by the language. 
	\item Exxceptions thrown by standard library Routines
	\item Excpetions thrown by user code, throw statement.
\end{itemize}

\subsection{Handling Exceptions}
\begin{itemize}
	\item Programmers can handle any exceptions they choose, this may be all excpetions, or some. 
	\item The things you can throw could be like int or char or sth, or it could even be some class object type. 
	\item You can also have as many try blocks as possible for each try block. 
	\item If not exception is thrown, then the exception handler is skipped. 
	\item An example of handling exceptions in C is: 
	\begin{lstlisting}[language=C++]
		try:
		{
			// risky stuff
		}
		catch (Formal Parameter)
		{
			// handle the exception
		}
		catch (Formal Parameter)
		{
			// Handle yet another exception
		}
		catch (DivideByZeroException ex)
		{
			cout<<ex.what()<<endl;
		}
	\end{lstlisting}
	\item There are 3 different kinds of syntaxes related to the throw keywordstyle
	\begin{lstlisting}
		throw(exception)
		throw exception.
		throw
	\end{lstlisting}
\end{itemize}


\end{document}

