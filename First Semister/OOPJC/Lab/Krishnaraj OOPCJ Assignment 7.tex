% This is a Basic Assignment Paper but with like Code and stuff allowed in it. 

\documentclass[11pt]{article}

% Preamble

\usepackage[margin=1in]{geometry}
\usepackage{amsfonts, amsmath, amssymb}
\usepackage{fancyhdr, float, graphicx}
\usepackage[utf8]{inputenc} % Required for inputting international characters
\usepackage[T1]{fontenc} % Output font encoding for international characters
\usepackage{fouriernc} % Use the New Century Schoolbook font
\usepackage[nottoc, notlot, notlof]{tocbibind}
\usepackage{listings}
\usepackage{xcolor}

\definecolor{codegreen}{rgb}{0,0.6,0}
\definecolor{codegray}{rgb}{0.5,0.5,0.5}
\definecolor{codepurple}{rgb}{0.58,0,0.82}
\definecolor{backcolour}{rgb}{0.95,0.95,0.92}

\lstdefinestyle{mystyle}{
    backgroundcolor=\color{backcolour},   
    commentstyle=\color{codegreen},
    keywordstyle=\color{magenta},
    numberstyle=\tiny\color{codegray},
    stringstyle=\color{codepurple},
    basicstyle=\ttfamily\footnotesize,
    breakatwhitespace=false,         
    breaklines=true,                 
    captionpos=b,                    
    keepspaces=true,                 
    numbers=left,                    
    numbersep=5pt,                  
    showspaces=false,                
    showstringspaces=false,
    showtabs=false,                  
    tabsize=2
}

\lstset{style=mystyle}

% Header and Footer
\pagestyle{fancy}
\fancyhead{}
\fancyfoot{}
\fancyhead[L]{\textit{\Large{OOPJC Assignment 7}}}
%\fancyhead[R]{\textit{something}}
\fancyfoot[C]{\thepage}
\renewcommand{\footrulewidth}{1pt}



% Other Doc Editing
% \parindent 0ex
%\renewcommand{\baselinestretch}{1.5}

\begin{document}

\begin{titlepage}
	\centering

	%---------------------------NAMES-------------------------------

	\huge\textsc{
		MIT World Peace University
	}\\

	\vspace{0.75\baselineskip} % space after Uni Name

	\LARGE{
		Object Oriented Programming with Java and C++\\
		Second Year B. Tech, Semester 1
	}

	\vfill % space after Sub Name

	%--------------------------TITLE-------------------------------

	\rule{\textwidth}{1.6pt}\vspace*{-\baselineskip}\vspace*{2pt}
	\rule{\textwidth}{0.6pt}
	\vspace{0.75\baselineskip} % Whitespace above the title



	\huge{\textsc{
			Multithreading using Thread Class and Runnable Interface in Java
		}} \\



	\vspace{0.5\baselineskip} % Whitespace below the title
	\rule{\textwidth}{0.6pt}\vspace*{-\baselineskip}\vspace*{2.8pt}
	\rule{\textwidth}{1.6pt}

	\vspace{1\baselineskip} % Whitespace after the title block

	%--------------------------SUBTITLE --------------------------	

	\LARGE\textsc{
		Practical Report\\
		Assignment 7
	} % Subtitle or further description
	\vfill

	%--------------------------AUTHOR-------------------------------

	Prepared By
	\vspace{0.5\baselineskip} % Whitespace before the editors

	\Large{
		Krishnaraj Thadesar \\
		Cyber Security and Forensics\\
		Batch A1, PA 20
	}


	\vspace{0.5\baselineskip} % Whitespace below the editor list
	\today

\end{titlepage}


\tableofcontents
\thispagestyle{empty}
\clearpage


\setcounter{page}{1}

\section{Aim and Objectives}
\subsection*{Aim}
Implementing Solutions on Multithreading using Thread Class and Runnable Interface
\subsection*{Objectives}
\begin{enumerate}
	\item To understand Multithreading in Java
	\item To learn two different ways to create threads in Java
\end{enumerate}
\section{Problem Statements}

\subsection{Problem 1 in Java}
Write a program to create a multithreaded calculator that does addition, subtraction,
multiplication, and division using separate threads.
Additionally also handle '/ by zero' exception by the division method.

\subsection{Problem 2 in Java}
Print even and odd numbers in increasing order using two threads in Java

\section{Theory}
% Multithreading in Java
% 
%  Life Cycle of a Thread
% 
%  Two ways to create a Thread
% 
%  How to perform multiple tasks by multiple threads
% 
%  Thread Scheduler
% 
%  Joining a thread
\section{Platform}
\textbf{Operating System}: Arch Linux x86-64 \\
\textbf{IDEs or Text Editors Used}: Visual Studio Code\\
\textbf{Compilers} : g++ and gcc on linux for C++, and javac, with JDK 18.0.2 for Java\\

\section{Input}
\subsection*{For Problem 1}
\begin{enumerate}
	\item 2 numbers
	\item Choice about what to do with those numbers 
\end{enumerate}
\subsection*{For Problem 2}
\begin{enumerate}
	\item The Maximum limit up to which the user wants to see the odd and even numbers printed
\end{enumerate}

\section{Output}
\subsection*{For Problem 1}
\begin{enumerate}
	\item Menu about what to do with numbers
	\item Output of the calculation done with those numbers
\end{enumerate}
\subsection*{For Problem 2}
\begin{enumerate}
	\item Even numebers and Odd numbers in Ascending order upto the specified limit. 
\end{enumerate}


\section{Code}

\subsection{Java Implementation of Problem 1}

\lstinputlisting[language=java, caption=Probelm 1.java]{../Programs/java_implementations/assignment_7/assignment_7_problem_1.java}

\subsubsection{Java Output}
\lstinputlisting[caption=Output for Problem 1 - calculations]{../Programs/java_implementations/assignment_7/problem_1_output.txt}

\subsection{Java Implementation of Problem 2}

\lstinputlisting[language=java, caption=Multithreading Even Odd]{../Programs/java_implementations/assignment_7/assignment_7_problem_2.java}

\subsubsection{Java Output}
\lstinputlisting[caption=Output for ProblemHillStation 2]{../Programs/java_implementations/assignment_7/problem_2_output.txt}


\section{Conclusion}
Thus, learnt the use of thread class in java and performed multithreading operations.

\pagebreak

\section{FAQs}

\begin{enumerate}
	\item \textit{Why do we use collection framework?}
	
	\item \textit{Which is best collection framework in Java?}
	
	\item \textit{What is difference between array and collection?}
	
	\item \textit{What is HashMap in Java?}
	
\end{enumerate}

\end{document}