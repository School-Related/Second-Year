% This is a Basic Assignment Paper but with like Code and stuff allowed in it. 

\documentclass[11pt]{article}

% Preamble

\usepackage[margin=1in]{geometry}
\usepackage{amsfonts, amsmath, amssymb}
\usepackage{fancyhdr, float, graphicx}
\usepackage[utf8]{inputenc} % Required for inputting international characters
\usepackage[T1]{fontenc} % Output font encoding for international characters
\usepackage{fouriernc} % Use the New Century Schoolbook font
\usepackage[nottoc, notlot, notlof]{tocbibind}
\usepackage{listings}
\usepackage{xcolor}
\usepackage{karnaugh-map}
\usepackage{pdfpages}

\definecolor{codegreen}{rgb}{0,0.6,0}
\definecolor{codegray}{rgb}{0.5,0.5,0.5}
\definecolor{codepurple}{rgb}{0.58,0,0.82}
\definecolor{backcolour}{rgb}{0.95,0.95,0.92}

\lstdefinestyle{mystyle}{
    backgroundcolor=\color{backcolour},   
    commentstyle=\color{codegreen},
    keywordstyle=\color{magenta},
    numberstyle=\tiny\color{codegray},
    stringstyle=\color{codepurple},
    basicstyle=\ttfamily\footnotesize,
    breakatwhitespace=false,         
    breaklines=true,                 
    captionpos=b,                    
    keepspaces=true,                 
    numbers=left,                    
    numbersep=5pt,                  
    showspaces=false,                
    showstringspaces=false,
    showtabs=false,                  
    tabsize=2
}

\lstset{style=mystyle}

% Header and Footer
\pagestyle{fancy}
\fancyhead{}
\fancyfoot{}
\fancyhead[L]{\textit{\Large{DECA Active Learning CCA 3}}}
%\fancyhead[R]{\textit{something}}
\fancyfoot[C]{\thepage}
\renewcommand{\footrulewidth}{1pt}



% Other Doc Editing
% \parindent 0ex
%\renewcommand{\baselinestretch}{1.5}

\begin{document}

\begin{titlepage}
	\centering

	%---------------------------NAMES-------------------------------

	\huge\textsc{
		MIT World Peace University
	}\\

	\vspace{0.75\baselineskip} % space after Uni Name

	\LARGE{
		Digital Electronics and Computer Architecture\\
		Second Year B. Tech, Semester 1
	}

	\vfill % space after Sub Name

	%--------------------------TITLE-------------------------------

	\rule{\textwidth}{1.6pt}\vspace*{-\baselineskip}\vspace*{2pt}
	\rule{\textwidth}{0.6pt}
	\vspace{0.75\baselineskip} % Whitespace above the title



	\huge{\textsc{
			Processor Structure, function and register organization\\
			And 1 Numerical.
		}} \\



	\vspace{0.5\baselineskip} % Whitespace below the title
	\rule{\textwidth}{0.6pt}\vspace*{-\baselineskip}\vspace*{2.8pt}
	\rule{\textwidth}{1.6pt}

	\vspace{1\baselineskip} % Whitespace after the title block

	%--------------------------SUBTITLE --------------------------	

	\LARGE\textsc{
		Practical Report
	} % Subtitle or further description
	\vfill

	%--------------------------AUTHOR-------------------------------

	Prepared By
	\vspace{0.5\baselineskip} % Whitespace before the editors

	\Large{
		Krishnaraj Thadesar, PA20 \\
		Atharva Yadav, PA24\\
		Anuj Choudhary, PA26\\
		Harshal Thoke, PA77\\
		Yukta Hande, PA99\\
		Shreny Jain, PA66\\
	}


	\vspace{0.5\baselineskip} % Whitespace below the editor list
	\today

\end{titlepage}


\tableofcontents
\thispagestyle{empty}
\clearpage


\setcounter{page}{1}
\section{Problem Statement}
\textit{Illustrate the Processor Structure, function and register organization using, the Mind Mapping concept.}

\section{Numerical}
Consider a computer system with byte addressable primary memory of size
$2^32$ bytes. It has direct mapped caches of size $32KB(1KB = 2^10 bytes)$,
and each cache block is 64 bytes. What is the size of the tag field bit?

\section{Theory}
\subsection{Processor Requirements}
\begin{enumerate}
	\item Requirements placed on the processor:
	\item Fetch instruction: reads an instruction from memory;
	\item Interpret instruction: determines what action to perform;
	\item Fetch data: if necessary read data from memory or an I/O module.
	\item Process data: If necessary perform arithmetic / logical operation on data.
	\item Write data: If necessary write data to memory or an I/O module.
\end{enumerate}

To do these things the processor needs to:
\begin{enumerate}
	\item
	      • Store some data temporarily
	      • Remember the location of the next instruction;
\end{enumerate}
While an instruction is being executed: In other words, the processor needs a small internal memory, called \textbf{Registers.}


\subsection{Major Parts of a CPU}
\begin{enumerate}
	\item Arithmetic and Logic Unit (ALU): Performs computation or processing of data
	\item Control Unit: Moves data and instructions in and out of the processor. Also controls the operation of the ALU;
	\item Registers: Used as internal memory;
	\item System Bus: Acting as a pathway between processor, memory and I/O module(s);
\end{enumerate}

\subsection{Register Organization}

\begin{enumerate}
	\item Registers in the processor perform two roles:
	\item User-visible registers:
	\item Used as internal memory by the assembly language programmer;
	\item Control registers: Used to control the operation of the processor;
	\item Status Register: Used to check the status of the processor or ALU;
\end{enumerate}

\subsection{User Visible Register}
\begin{enumerate}
	\item May be referenced by the programmer, categorized into:
	\item General purpose
	\item Data
	\item Address
	\item Condition codes
\end{enumerate}

\subsection{General Purpose Registers}
\begin{enumerate}
	\item Can be assigned to a variety of functions by the programmer:
	\item Memory reference and backup;
	\item Register reference and backup;
	\item Data reference and backup;
	\item These are the ones you use in the laboratory

\end{enumerate}


\subsection{Data Registers}
\begin{enumerate}
	\item May be used only to hold data and cannot hold addresses:
	\item Must be able to hold values of most data types;
	\item Some machines allow two contiguous registers to be used:
	\item For holding double-length values

\end{enumerate}

\subsection{Flags or Condition Code Bits}
\begin{enumerate}
    \item Condition code bits are collected into one or more control register
    \item Interruption results in all user-visible registers being saved;
    \item These are then restored on return;
    \item Allows each subroutine to use the user-visible registers independently;
\end{enumerate}

\subsection{Examples of Control Registers}
\begin{enumerate}
    \item Program counter (PC): Contains instruction address to be fetched;
    \item Instruction Register (IR): Contains last instruction fetched;
    \item Memory address register (MAR): Contains memory location address;
    \item Memory buffer register (MBR): Contains:
    \item Word of data to be written to memory;
    \item Word of data read from memory.
\end{enumerate}

\subsection{PSW Register}
\begin{enumerate}
    \item Many processors include a program status word (PSW) register:
    \item Contains condition codes plus other status information
    \item Common fields or flags include the following:
    \item Sign: Sign bit of the result of the last arithmetic operation;
    \item Zero: when the result is 0;
    \item Carry: Set if an operation resulted in a carry/borrow bit;
    \item Equal: Set if a logical compare result is equality.
    \item Overflow: Used to indicate arithmetic overflow.
    \item Interrupt Enable/Disable: Used to enable or disable interrupts.
\end{enumerate}


\section{Numerical}

\textbf{Concept}\\
Memory address $=2^{32}$ bytes\\
Data Cache $=32 \mathrm{~KB}=2^{15} \mathrm{~B}$\\
Block Size $=64$ bytes $=2^6 \mathrm{~B}$\\
\textbf{Concept}:\\
In a direct Mapped, main memory can be represented as\\
\begin{tabular}{|l|l|l|}
\hline Tag & Lines & Block offset \\
\hline
\end{tabular}\\
$32=\operatorname{tag}+$ lines $+$ Block offset (In bits)\\
Calculation:\\
Assume byte addressable:\\
number of lines in cache memory\\
$$=\frac{2^{15}}{2^6}=2^9$$
$$32=\operatorname{tag}+9+6$$
$\therefore$ tag $=17$
bits

\section{Prodecure}
\begin{enumerate}
	\item Draw Flowcharts, Mind maps and Diagrams on Paper
	\item Try to draw them on Online collaborative websites like draw.io and excalidraw.com
	\item Solve the Numerical
	\item Combine pdfs into 1.
\end{enumerate}


\section{Conclusion}
Thus learnt more about processor structure, architecture, and Register Types. Also learnt to solve numericals based on Computer architecture.

\end{document}

