% This is a Basic Assignment Paper but with like Code and stuff allowed in it, there is also url, hyperlinks from contents included. 

\documentclass[11pt]{article}

% Preamble

\usepackage[margin=1in]{geometry}
\usepackage{amsfonts, amsmath, amssymb}
\usepackage{fancyhdr, float, graphicx}
\usepackage[utf8]{inputenc} % Required for inputting international characters
\usepackage[T1]{fontenc} % Output font encoding for international characters
\usepackage{fouriernc} % Use the New Century Schoolbook font
\usepackage[nottoc, notlot, notlof]{tocbibind}
\usepackage{listings}
\usepackage{xcolor}
\usepackage{blindtext}
\usepackage{hyperref}
\hypersetup{
    colorlinks=true,
    linkcolor=black,
    filecolor=magenta,      
    urlcolor=cyan,
    pdfpagemode=FullScreen,
    }

\definecolor{codegreen}{rgb}{0,0.6,0}
\definecolor{codegray}{rgb}{0.5,0.5,0.5}
\definecolor{codepurple}{rgb}{0.58,0,0.82}
\definecolor{backcolour}{rgb}{0.95,0.95,0.92}

\lstdefinestyle{mystyle}{
    backgroundcolor=\color{backcolour},   
    commentstyle=\color{codegreen},
    keywordstyle=\color{magenta},
    numberstyle=\tiny\color{codegray},
    stringstyle=\color{codepurple},
    basicstyle=\ttfamily\footnotesize,
    breakatwhitespace=false,         
    breaklines=true,                 
    captionpos=b,                    
    keepspaces=true,                 
    numbers=left,                    
    numbersep=5pt,                  
    showspaces=false,                
    showstringspaces=false,
    showtabs=false,                  
    tabsize=2
}

\lstset{style=mystyle}

% Header and Footer
\pagestyle{fancy}
\fancyhead{}
\fancyfoot{}
\fancyhead[L]{\textit{\Large{OOPJC Mini Project Report}}}
%\fancyhead[R]{\textit{something}}
\fancyfoot[C]{\thepage}
\renewcommand{\footrulewidth}{1pt}



% Other Doc Editing
% \parindent 0ex
%\renewcommand{\baselinestretch}{1.5}

\begin{document}

\begin{titlepage}
	\centering

	%---------------------------NAMES-------------------------------

	\huge\textsc{
		MIT World Peace University
	}\\

	\vspace{0.75\baselineskip} % space after Uni Name

	\LARGE{
		Indian Constitution\\
		Second Year B. Tech, Semester 1
	}

	\vfill % space after Sub Name

	%--------------------------TITLE-------------------------------

	\rule{\textwidth}{1.6pt}\vspace*{-\baselineskip}\vspace*{2pt}
	\rule{\textwidth}{0.6pt}
	\vspace{0.75\baselineskip} % Whitespace above the title



	\huge{\textsc{
			Criminal Laws in India
		}} \\



	\vspace{0.5\baselineskip} % Whitespace below the title
	\rule{\textwidth}{0.6pt}\vspace*{-\baselineskip}\vspace*{2.8pt}
	\rule{\textwidth}{1.6pt}

	\vspace{1\baselineskip} % Whitespace after the title block

	%--------------------------SUBTITLE --------------------------	

	\LARGE\textsc{
		Theory Assignment
	} % Subtitle or further description
	\vfill

	%--------------------------AUTHOR-------------------------------

	Prepared By
	\vspace{0.5\baselineskip} % Whitespace before the editors

	\Large{
		Krishnaraj Thadesar \\
		Cyber Security and Forensics\\
		Batch A1, PA 20
	}


	\vspace{0.5\baselineskip} % Whitespace below the editor list
	\today

\end{titlepage}


\tableofcontents
\thispagestyle{empty}
\clearpage

\setcounter{page}{1}

\section{Basic Definitons}

\subsection{What is a Crime?}

\begin{quote}
	\textit{A crime is an act committed or omitted in violation of a public law forbidding or commanding it. The term crime does not, in modern criminal law, have any simple and universally accepted definition, though statutory definitions have been provided for certain purposes. The most popular view is that crime is a category created by law; in other words, something is a crime if declared as such by the relevant and applicable law.}
\end{quote}

\subsection{What is a Criminal?}

\begin{quote}
	\textit{A criminal is a person who has committed a crime. The term "criminal" does not have any simple and universally accepted definition, and the legal meaning of the term varies between jurisdictions. The most popular view is that a criminal is a person who has been charged with a crime by the relevant and applicable law.}
\end{quote}

\section{What is Criminal Law?}

Criminal law is the body of law that relates to crime. It proscribes conduct perceived as threatening, harmful, or otherwise endangering to the property, health, safety, and moral welfare of people inclusive of one's self. Most criminal law is established by statute, which is to say that the laws are enacted by a legislature. Criminal law includes the punishment and rehabilitation of people who violate such laws.

\subsection{Criminal Law vs from Civil Law?}

Criminal law differs from civil law, where emphasis is more on dispute resolution and victim compensation than on punishment or rehabilitation. Criminal law is in force in most countries at the national or federal level, and may also be in force at lower levels, such as state or provincial. Criminal law is distinct from the law of civil wrongs, which is the body of law that deals with civil wrongs, including negligence, breach of contract, and other wrongs for which monetary compensation may be awarded.

\subsection{Criminal Procedure Law and Substantive Law?}

Criminal law is sometimes classified as substantive law, as opposed to procedural law. The latter may be defined as the rules that govern the investigation, prosecution, trial, and punishment of persons accused of breaking the substantive criminal laws. Criminal procedure is a branch of the larger field of procedural law. Criminal procedure is the body of law that sets out the rules and standards that courts follow when hearing criminal cases. Criminal procedure is sometimes called "criminal justice procedure" or "criminal litigation procedure."

\subsection{Two Branches of Criminal Law}

Criminal law is sometimes divided into two main branches: criminal law proper and regulatory offences. Criminal law proper consists of all the crimes that are defined by the common law or by statute. Regulatory offences are those that are created by statute and that regulate certain activities, such as the operation of a business or the possession of a firearm. Regulatory offences are sometimes called "quasi-criminal offences" or "quasi-criminal offences."

\section{Criminal Law in India}

\subsection{History of Criminal Law in India}

\textit{The history of codification of modern criminal law in India generally begins from the advent of the British rule. However, its roots date back to the Vedic age and the rule of various Hindu and Muslim dynasties. The modern criminal justice system is based on English laws and practices. These practices are practical as well as contemporary. As a result, a major chunk of criminal laws that exist today still relies on the British-era laws.}\\

In ancient India, Hindu religious laws contained many provisions for governing criminal as well as civil matters. The Vedas, Shrutis, Smritis and even other documents like Manusmriti contain provisions regulating criminal law. The practice of codifying criminal offences existed in this period as well.\\

These laws also contained detailed procedural rules and regulations for trials. There are some records which also show the existence of principles of evidence to govern these trials.\\

When Warren Hastings introduced his Judicial Plan of 1772, he did not many any severe changes to substantive criminal law.  In 1773, he slowly started changing rules of procedure and evidence in existing criminal laws. For example, he abolished the practice of allowing male relates of victims to pardon their killers. During this time, serious offences like homicide became crimes against the state instead of being private offences. This laid the foundation of the modern practice of the state prosecuting people who commit public offences.\\

From 1790 onwards, Lord Cornwallis extended the process of codifying criminal law. Major changes took place in the subject of sentencing. As a result, the process of levying punishments physically harming and dismembering convicts slowly started fading. Lord Wellesley made even more changes to the offences of murder and homicide in the early 1800s. For example, the law now made distinctions between intentional and unintentional killing.
Furthermore, rules of evidence became stricter and the threshold of proof to indicate guilt increased greatly. In presidency towns like Madras, Bombay and Calcutta, the British made many changes keeping local conditions in mind.\\

\subsection{The Indian Penal Code}

The Indian Penal Code (IPC) is the main criminal code of India. It was enacted in 1860, and was based on the model of the British Penal Code of 1857. The IPC is a comprehensive code that covers all substantive aspects of criminal law. The IPC is divided into 23 parts, with 511 sections. The IPC is supplemented by the Indian Evidence Act, 1872, and the Code of Criminal Procedure, 1973.

\subsection{The Indian Evidence Act}

The Indian Evidence Act, 1872, is the main evidence law of India. It was enacted in 1872, and was based on the model of the British Evidence Act of 1872. The Indian Evidence Act is a comprehensive code that covers all substantive aspects of evidence law. The Indian Evidence Act is divided into 13 parts, with 114 sections. The Indian Evidence Act is supplemented by the Indian Penal Code, 1860, and the Code of Criminal Procedure, 1973.

\subsection{The Code of Criminal Procedure}

The Code of Criminal Procedure, 1973, is the main criminal procedure law of India. It was enacted in 1973, and was based on the model of the British Criminal Procedure Act of 1973. The Code of Criminal Procedure is a comprehensive code that covers all substantive aspects of criminal procedure. The Code of Criminal Procedure is divided into 25 parts, with 511 sections. The Code of Criminal Procedure is supplemented by the Indian Penal Code, 1860, and the Indian Evidence Act, 1872.


\section{Criminal Law Articles in the Indian Constitution}


\subsection{Article 20}
\begin{quote}
	20(1): No person shall be convicted of any offence except for violation of a law in force at the time of the commission of the Act charged as an offence, nor be subjected to a penalty greater than that which might have been inflicted under the law in force at the time of the commission of the offence.\\
	20(2): No person shall be prosecuted and punished for the same offence more than once.\\
	20(3): No person accused of any offence shall be compelled to be a witness against himself.\\
\end{quote}

\subsection{Article 22}
\begin{quotation}
	Article 22 of the Indian Constitution provides that no person who is arrested shall be detained in custody without being informed, as soon as may be, of the grounds for such arrest nor shall he be denied the right to consult, and to be defended by, a legal practitioner of his choice.
\end{quotation}

\subsection{The subjects relating to the criminal justice system as included in the Seventh Schedule of the Constitution of India are given below}

\textbf{Central List}
\begin{itemize}
	\item Central Bureau of Intelligence and Investigation.
	\item Preventive detention for reasons connected with Defence, Foreign Affairs, or the security of India; persons subjected to such detention.
	\item Constitution, organization, jurisdiction and powers of the Supreme Court (including contempt of such Court) and fees taken therein; persons entitled to practice before the Supreme Court.
	\item Constitution and organization including vacations of the High Courts except provisions as to officers and servants of High Courts; persons entitled to practice before the High Courts.
	\item Extension of the jurisdiction of a High Court to, and exclusion of the jurisdiction of a High Court from, any Union Territory.
	\item Extension of the powers and jurisdiction of members of a police force belonging to any state to any area outside that state, but not so as to enable the police of one state to exercise powers and jurisdiction in any area outside that state without the consent of the government of the state in which such area is situated; extension of the powers and jurisdiction of members of a police force belonging to any state to railway areas outside that state.
	\item Offences against laws with respect to any of the matters in this List.
	\item Jurisdiction and powers of all courts, except the Supreme Court, with respect to any of the matters in this list; admiralty jurisdiction.

\end{itemize}

\textbf{State List}

\begin{itemize}
	\item
	\item Public order but not including the use of any naval, military or air force or any other armed force of the Union or any other force subject to the control of the Union or any contingent or unit thereof in aid of the civil power.
	\item Police including railway and village police subject to the provisions of entry 2A of List-I
	\item Officers and servants of the High Court; procedure in rent and revenue courts; fees taken in all courts except the Supreme Court.
	\item Prisons, reformatories, Borstal institutions and institutions of a like nature and persons detained therein; arrangements with other states for the use of prisons and other institutions.
	\item Offences against laws with respect to any of the matters in this List.
	\item Jurisdiction and powers of all courts, except the Supreme Court, with respect to any of the matters in this List.

\end{itemize}

\textbf{Concurrent List}

\begin{itemize}

	\item Criminal law, including all matters included in the Indian Penal Code at the commencement of this Constitution but excluding offences against laws with respect to any of the matters specified in List I or List II and excluding the use of naval, military or air forces or any other armed forces of the Union in aid of the civil power.
	\item Criminal procedure, including all matters included in the Code of Criminal Procedure at the commencement of this Constitution.
	\item Preventive detention for reasons connected with the security of a state, the maintenance of public order, or the maintenance of supplies and services essential to the community; persons subjected to such detention.
	\item Removal from one state to another state of prisoners, accused persons and persons subjected to preventive detention for reasons specified in entry 3 of this List.
	\item Administration of justice; constitution and organization of all courts, except the Supreme Court and the High Courts.
	\item Evidence and oaths; recognition of laws, public acts and records, and judicial proceedings.
	\item Legal, medical and other professions.
	\item Jurisdiction and powers of all courts, except the Supreme Court with respect to any of the matters in this List.

\end{itemize}

\section{Conclusion}
From the foregoing analysis of the constitutional provisions, it is observed that the ideals of equality, liberty and dignity of the individual were kept constantly in view while framing the Constitution. The framers gave top priority to justice. They made several provisions for criminal justice and its administration in the Constitution itself. While recognizing rights of the people, the imperatives of security, unity and integrity of the State were also kept in view constantly. The Constitution allows the State, which includes the police and magistracy, to impose reasonable restrictions on some of the Fundamental Rights of the people in certain circumstances to maintain order, decency, morality, etc.\\

Thus, the Constitution contains adequate provisions for fair administration of criminal justice. However, for achieving desired results, the provisions need to be implemented meticulously obeying the spirit behind them and not just the letter. It may be proper to recall the following remarks of Dr. Rajendra Prasad, the first President of India. While speaking in the Constituent Assembly on

26th November 1949, he said:
\begin{quote}
	\textit{"If the people who are elected are capable and men of character and integrity, they would be able to make the best even of a defective Constitution. If they are lacking in these, the Constitution cannot help the country."}
\end{quote}

\end{document}