% This is a Basic Assignment Paper but with like Code and stuff allowed in it, there is also url, hyperlinks from contents included. 

\documentclass[11pt]{article}

% Preamble

\usepackage[margin=1in]{geometry}
\usepackage{amsfonts, amsmath, amssymb}
\usepackage{fancyhdr, float, graphicx}
\usepackage[utf8]{inputenc} % Required for inputting international characters
\usepackage[T1]{fontenc} % Output font encoding for international characters
\usepackage{fouriernc} % Use the New Century Schoolbook font
\usepackage[nottoc, notlot, notlof]{tocbibind}
\usepackage{listings}
\usepackage{xcolor}
\usepackage{blindtext}
\usepackage{hyperref}
\hypersetup{
    colorlinks=true,
    linkcolor=black,
    filecolor=magenta,      
    urlcolor=cyan,
    pdfpagemode=FullScreen,
    }

\definecolor{codegreen}{rgb}{0,0.6,0}
\definecolor{codegray}{rgb}{0.5,0.5,0.5}
\definecolor{codepurple}{rgb}{0.58,0,0.82}
\definecolor{backcolour}{rgb}{0.95,0.95,0.92}

\lstdefinestyle{mystyle}{
    backgroundcolor=\color{backcolour},   
    commentstyle=\color{codegreen},
    keywordstyle=\color{magenta},
    numberstyle=\tiny\color{codegray},
    stringstyle=\color{codepurple},
    basicstyle=\ttfamily\footnotesize,
    breakatwhitespace=false,         
    breaklines=true,                 
    captionpos=b,                    
    keepspaces=true,                 
    numbers=left,                    
    numbersep=5pt,                  
    showspaces=false,                
    showstringspaces=false,
    showtabs=false,                  
    tabsize=2
}

\lstset{style=mystyle}

% Header and Footer
\pagestyle{fancy}
\fancyhead{}
\fancyfoot{}
\fancyhead[L]{\textit{\Large{Database Management Systems Assignment 4}}}
%\fancyhead[R]{\textit{something}}
\fancyfoot[C]{\thepage}
\renewcommand{\footrulewidth}{1pt}



% Other Doc Editing
% \parindent 0ex
%\renewcommand{\baselinestretch}{1.5}

\begin{document}

\begin{titlepage}
	\centering

	%---------------------------NAMES-------------------------------

	\huge\textsc{
		MIT World Peace University
	}\\

	\vspace{0.75\baselineskip} % space after Uni Name

	\LARGE{
		Database Management Systems\\
		Second Year B. Tech, Semester 4
	}

	\vfill % space after Sub Name

	%--------------------------TITLE-------------------------------

	\rule{\textwidth}{1.6pt}\vspace*{-\baselineskip}\vspace*{2pt}
	\rule{\textwidth}{0.6pt}
	\vspace{0.75\baselineskip} % Whitespace above the title



	\huge{\textsc{
			Group Functions, Join and Nested Queries.
		}} \\



	\vspace{0.5\baselineskip} % Whitespace below the title
	\rule{\textwidth}{0.6pt}\vspace*{-\baselineskip}\vspace*{2.8pt}
	\rule{\textwidth}{1.6pt}

	\vspace{1\baselineskip} % Whitespace after the title block

	%--------------------------SUBTITLE --------------------------	

	\LARGE\textsc{
		Assignment No. 4
	} % Subtitle or further description
	\vfill

	%--------------------------AUTHOR-------------------------------

	Prepared By
	\vspace{0.5\baselineskip} % Whitespace before the editors

	\Large{
		Krishnaraj Thadesar \\
		Cyber Security and Forensics\\
		Batch A1, PA 20
	}


	\vspace{0.5\baselineskip} % Whitespace below the editor list
	\today

\end{titlepage}


\tableofcontents
\thispagestyle{empty}
\clearpage

\setcounter{page}{1}

\section{Aim}
Write suitable select command to get requested data from tables

\section{Objectives}
\begin{enumerate}
	\item To study Subqueries, Group, Joins and Views
\end{enumerate}

\section{Problem Statement}
Create tables and solve given queries using , Group, Joins and Views

\section{Theory}

% Theory:
% Explain SQL Join types
% Points to be included in
% Syntax:
% Description:
% Example:
% [Include Table specifying the logical operators to be used ]
% Explain Joins, Types of Joins
% Syntax:
% Description:
% Example:
% Input: Database
% Output: Data as per request
% Platform:
% Conclusion: 
% FAQs:






\section{Platform}
\textbf{Operating System}: Arch Linux x86-64 \\
\textbf{IDEs or Text Editors Used}: Draw.io for Drawing the ER diagram.

% \section{Pseudo Code or Algorithm}

\section{Input}
Given Database from the Problem Statement for the Assignment for our batch. (A1 PA 20)

\section{Executed Queries}

\subsection{Questions SetA}

\begin{lstlisting}[language=SQL]

\end{lstlisting}

\subsection{Questions Set B}

\begin{lstlisting}[language=SQL]
	
\end{lstlisting}
\section{Conclusion}
Thus, we have learned to Select Group By, Joins and Subqueries commands thoroughly.
\clearpage

\section{FAQ}
\begin{enumerate}
	\item \textbf{When to use self join? How does it differ from other joins?}\\

	      A self join is used when you need to join a table with itself, typically to find relationships between rows in the same table. It differs from other joins in that you are joining a table with itself rather than joining two separate tables. A self join can be performed using an alias to distinguish between the two copies of the table being joined.


	      \begin{lstlisting}[language=sql]
SELECT t1.employee_name, t2.employee_name 
FROM employees t1 
JOIN employees t2 ON t1.manager_id = t2.employee_id;
\end{lstlisting}

	\item \textbf{Compare Cross Join with Natural Join. Share your comments.}\\

	      A cross join produces the Cartesian product of two tables, resulting in a combination of all rows from one table with all rows from another table. A natural join matches two tables based on their common column names. It automatically eliminates duplicate columns from the result set, and the result set only contains the columns with the same name from both tables.

	      \begin{lstlisting}[language=sql]
	SELECT *
	FROM table1
	CROSS JOIN table2;
\end{lstlisting}

	\item \textbf{What is the importance of SQL joins in database management? Explain its types.}\\

	      SQL joins are important in database management because they allow you to combine data from two or more tables into a single result set. This allows you to extract meaningful information from your data by revealing relationships between tables. There are four main types of SQL joins: inner join, left join, right join, and full outer join.


	\item \textbf{What are the different types of Joins in SQL?}\\

	      The different types of SQL joins are:

	      \begin{itemize}
		      \item Inner join: returns only the matching rows from both tables based on the specified join condition.

		      \item Left join: returns all the rows from the left table and the matching rows from the right table based on the specified join condition.

		      \item Right join: returns all the rows from the right table and the matching rows from the left table based on the specified join condition.

		      \item Full outer join: returns all the rows from both tables, matching rows where possible and filling in NULL values for non-matching rows.

	      \end{itemize}

	      \begin{lstlisting}[language=sql]
SELECT *
FROM table1
INNER JOIN table2
ON table1.column = table2.column;
	\end{lstlisting}

	      \begin{lstlisting}[language=sql]
SELECT *
FROM table1
LEFT JOIN table2
ON table1.column = table2.column;
\end{lstlisting}



	\item \textbf{State the difference between inner join and left join.}\\

	      The main difference between an inner join and a left join is that an inner join only returns matching rows from both tables based on the specified join condition, while a left join returns all the rows from the left table and the matching rows from the right table based on the specified join condition.

	      \begin{lstlisting}[language=sql]
SELECT *
FROM table1
LEFT JOIN table2
ON table1.column = table2.column;
\end{lstlisting}

	      \begin{lstlisting}[language=sql]
SELECT *
FROM table1
INNER JOIN table2
ON table1.column = table2.column;
\end{lstlisting}

	\item \textbf{State difference between left join and right join.}\\

	      The main difference between a left join and a right join is that a left join returns all the rows from the left table and the matching rows from the right table based on the specified join condition, while a right join returns all the rows from the right table and the matching rows from the left table based on the specified join condition.

	      \begin{lstlisting}[language=sql]
SELECT *
FROM table1
LEFT JOIN table2
ON table1.column = table2.column;
\end{lstlisting}

	      \begin{lstlisting}[language=sql]
SELECT *
FROM table1
RIGHT JOIN table2
ON table1.column = table2.column;
	\end{lstlisting}

\end{enumerate}


\end{document}