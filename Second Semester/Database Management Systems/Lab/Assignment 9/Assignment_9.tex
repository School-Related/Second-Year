% This is a Basic Assignment Paper but with like Code and stuff allowed in it, there is also url, hyperlinks from contents included. 

\documentclass[11pt]{article}

% Preamble

\usepackage[margin=1in]{geometry}
\usepackage{amsfonts, amsmath, amssymb}
\usepackage{fancyhdr, float, graphicx}
\usepackage[utf8]{inputenc} % Required for inputting international characters
\usepackage[T1]{fontenc} % Output font encoding for international characters
\usepackage{fouriernc} % Use the New Century Schoolbook font
\usepackage[nottoc, notlot, notlof]{tocbibind}
\usepackage{listings}
\usepackage{xcolor}
\usepackage{blindtext}
\usepackage{hyperref}
\hypersetup{
    colorlinks=true,
    linkcolor=black,
    filecolor=magenta,      
    urlcolor=cyan,
    pdfpagemode=FullScreen,
    }

\definecolor{codegreen}{rgb}{0,0.6,0}
\definecolor{codegray}{rgb}{0.5,0.5,0.5}
\definecolor{codepurple}{rgb}{0.58,0,0.82}
\definecolor{backcolour}{rgb}{0.95,0.95,0.92}

\lstdefinestyle{mystyle}{
    backgroundcolor=\color{backcolour},   
    commentstyle=\color{codegreen},
    keywordstyle=\color{magenta},
    numberstyle=\tiny\color{codegray},
    stringstyle=\color{codepurple},
    basicstyle=\ttfamily\footnotesize,
    breakatwhitespace=false,         
    breaklines=true,                 
    captionpos=b,                    
    keepspaces=true,                 
    numbers=left,                    
    numbersep=5pt,                  
    showspaces=false,                
    showstringspaces=false,
    showtabs=false,                  
    tabsize=2
}

\lstset{style=mystyle}

% Header and Footer
\pagestyle{fancy}
\fancyhead{}
\fancyfoot{}
\fancyhead[L]{\textit{\Large{Database Management Systems Assignment 9}}}
%\fancyhead[R]{\textit{something}}
\fancyfoot[C]{\thepage}
\renewcommand{\footrulewidth}{1pt}



% Other Doc Editing
% \parindent 0ex
%\renewcommand{\baselinestretch}{1.5}

\begin{document}

\begin{titlepage}
	\centering

	%---------------------------NAMES-------------------------------

	\huge\textsc{
		MIT World Peace University
	}\\

	\vspace{0.75\baselineskip} % space after Uni Name

	\LARGE{
		Database Management Systems\\
		Second Year B. Tech, Semester 4
	}

	\vfill % space after Sub Name

	%--------------------------TITLE-------------------------------

	\rule{\textwidth}{1.6pt}\vspace*{-\baselineskip}\vspace*{2pt}
	\rule{\textwidth}{0.6pt}
	\vspace{0.75\baselineskip} % Whitespace above the title



	\huge{\textsc{
        SQL Queries on Functions, Data Sorting, Subquery, Group by, Having, Set Operations and View
		}} \\



	\vspace{0.5\baselineskip} % Whitespace below the title
	\rule{\textwidth}{0.6pt}\vspace*{-\baselineskip}\vspace*{2.8pt}
	\rule{\textwidth}{1.6pt}

	\vspace{1\baselineskip} % Whitespace after the title block

	%--------------------------SUBTITLE --------------------------	

	\LARGE\textsc{
		Assignment No. 5
	} % Subtitle or further description
	\vfill

	%--------------------------AUTHOR-------------------------------

	Prepared By
	\vspace{0.5\baselineskip} % Whitespace before the editors

	\Large{
		Krishnaraj Thadesar \\
		Cyber Security and Forensics\\
		Batch A1, PA 20
	}


	\vspace{0.5\baselineskip} % Whitespace below the editor list
	\today

\end{titlepage}


\tableofcontents
\thispagestyle{empty}
\clearpage

\setcounter{page}{1}

\section{Aim}
Write suitable select commands to execute queries on the given data set.

\section{Objectives}
\begin{enumerate}
    \item To get basic understanding of Aggregate Functions, Order By clause
    \item To get basic understanding of Subquery or Inner query or Nested query and Select using subquery.
    \item To understand the basic concept of Correlated Subquery.
    \item To get familiar with the basic ALL, ANY, EXISTS, SOME functionality.
    \item To understand basic TCL commands
\end{enumerate}


\section{Problem Statement}
Create tables and solve given queries using Subqueries

\section{Theory}

\subsection{Aggregate Functions}

Aggregate functions are used to perform calculations on a set of values and return a single value. The following are the aggregate functions:

\begin{itemize}
    \item COUNT() - Returns the number of rows that matches a specified criteria
    \item SUM() - Returns the sum of all the values in a column
    \item AVG() - Returns the average value of a numeric column
    \item MIN() - Returns the smallest value of the selected column
    \item MAX() - Returns the largest value of the selected column
    
\end{itemize}

\subsection{Order By Clause}

The ORDER BY clause is used to sort the result-set in ascending or descending order. The ORDER BY keyword sorts the records in ascending order by default. To sort the records in descending order, use the DESC keyword.

\subsubsection*{Syntax}

\begin{lstlisting}[language=sql]
SELECT column_name(s)
FROM table_name
ORDER BY column_name(s) ASC|DESC;
\end{lstlisting}

\subsection{Group By Clause}

The GROUP BY clause is often used with aggregate functions (COUNT, MAX, MIN, SUM, AVG) to group the result-set by one or more columns.

\subsubsection*{Syntax}

\begin{lstlisting}[language=sql]
SELECT column_name, aggregate_function(column_name)
FROM table_name
WHERE column_name operator value
GROUP BY column_name;
\end{lstlisting}

\subsection{Subqueries}

A subquery (sub-select) is a query within a query. The subquery is executed first, and the main query uses the subquery as a source of data.

\subsubsection*{Syntax}

\begin{lstlisting}[language=sql]
SELECT column_name(s)
FROM table_name
WHERE column_name operator
(SELECT STATEMENT);
\end{lstlisting}

\subsection{Views}

A view is a virtual table based on the result-set of an SQL statement.

\subsubsection*{Syntax}

\begin{lstlisting}[language=sql]
CREATE VIEW view_name AS
SELECT column_name(s)
FROM table_name
WHERE condition;
\end{lstlisting}

\subsection{TCL Commands}

TCL stands for Transaction Control Language. It is a set of SQL commands that are used to control the transaction. The following are the TCL commands:

\begin{itemize}
    \item COMMIT - permanently saves all changes made by the transaction. 
    \item ROLLBACK - cancels all changes made by the transaction
    \item SAVEPOINT - sets a savepoint within a transaction
    \item SET TRANSACTION - sets the transaction characteristics for the current transaction
\end{itemize}

\section{Platform}
\textbf{Operating System}: Arch Linux x86-64 \\
\textbf{IDEs or Text Editors Used}: Draw.io for Drawing the ER diagram. \\

% \section{Pseudo Code or Algorithm}

\section{Input}
Given Database from the Problem Statement for the Assignment for our batch. (A1 PA 20)
\section{Creation and Insertion of Values in the Tables}

\lstinputlisting[language=SQL]{../../Programs/Assignment_5_queries.sql}

\section{Tables}

\begin{lstlisting}[language=sql]
MariaDB [dbms_lab]> select * from passenger;
+----------------+------------+----------+
| email          | first_name | surname  |
+----------------+------------+----------+
| beck@gmail.com | Gwen       | Beck     |
| joe@gmail.com  | Joe        | Goldberg |
| love@gmail.com | Love       | Quinn    |
+----------------+------------+----------+
3 rows in set (0.001 sec)

MariaDB [dbms_lab]> select * from airplane;
+--------+----------+----------+---------------+
| reg_no | model_no | capacity | name          |
+--------+----------+----------+---------------+
|    111 |        7 |      180 | Qatar Airways |
|    112 |        7 |      169 | Qatar Airways |
|    113 |        8 |      200 | Qatar Airways |
|    221 |       17 |      150 | Emirates      |
|    222 |       17 |      140 | Emirates      |
|    223 |       18 |      175 | Emirates      |
|    333 |       27 |      200 | Air India     |
|    334 |       27 |      150 | Air India     |
|    335 |       28 |      175 | Air India     |
+--------+----------+----------+---------------+
9 rows in set (0.001 sec)

MariaDB [dbms_lab]> select * from airline;
+---------------+
| name          |
+---------------+
| Air India     |
| Emirates      |
| Qatar Airways |
+---------------+
3 rows in set (0.001 sec)

MariaDB [dbms_lab]> select * from flights;
+-----------+------------+-----------+----------------+----------------+--------------+--------------+--------+
| flight_no | place_from | place_to  | departure_date | departure_time | arrival_date | arrival_time | reg_no |
+-----------+------------+-----------+----------------+----------------+--------------+--------------+--------+
|     12345 | Mumbai     | London    | 2021-07-27     | 12:12:12       | 2021-07-28   | 23:59:56     |    111 |
|     23456 | London     | Pune      | 2021-07-27     | 12:12:12       | 2021-07-28   | 22:59:56     |    333 |
|     67890 | Pune       | Bangalore | 2021-07-27     | 12:12:12       | 2021-07-27   | 16:59:56     |    221 |
+-----------+------------+-----------+----------------+----------------+--------------+--------------+--------+
3 rows in set (0.001 sec)

MariaDB [dbms_lab]> select * from flight_booking;
+----------------+-----------+----------+
| email          | flight_no | no_seats |
+----------------+-----------+----------+
| beck@gmail.com |     67890 |        6 |
| joe@gmail.com  |     23456 |        2 |
| love@gmail.com |     12345 |        6 |
+----------------+-----------+----------+
3 rows in set (0.001 sec)

\end{lstlisting}

\section{Queries}
\lstinputlisting[language=SQL]{../../Programs/Assignment_5.md}


\section{Conclusion}
Thus, we have learned Subqueries commands thoroughly.

\clearpage

\section{FAQ}
\begin{enumerate}

\item \textit{Explain following types of subqueries}
\begin{itemize}
    \item \textit{Single-row subquery}    
    \item \textit{Multiple-row subquery}
    \item \textit{Multiple-column subquery}
\end{itemize}

The Given Subqueries can be explained as such: 
\begin{itemize}
    \item \textit{Single-row subquery} : A subquery that returns a single row is called a single-row subquery. A special case is the scalar subquery, which returns a single row with one column. Scalar subqueries are acceptable (and often very useful) in virtually any situation where you could use a literal value, a constant, or an expression.
    \item \textit{Multiple-row subquery} : A subquery that returns multiple rows is called a multiple-row subquery. These queries are commonly used to generate result sets that will be passed to a DML or SELECT statement for further processing. Both single-row and multiple-row subqueries will be evaluated once, before the parent query is run. Single- and multiple-row subqueries can be used in the WHERE and HAVING clauses of the parent query, but there are restrictions on the legal comparison operators
    \item \textit{Multiple-column subquery} : A subquery that returns multiple columns is called a multiple-column subquery. It is also called a correlated subquery.    
    A correlated subquery has a more complex method of execution than single- and multiple-row subqueries and is potentially much more powerful. If a subquery references columns in the parent query, then its result will be dependent on the parent query. This makes it impossible to evaluate the subquery before evaluating the parent query.
\end{itemize}

\item \textit{When subquery is used?}

\begin{quotation}
Subqueries are queires within queries. 
\end{quotation}

They are used for the following purposes:
\begin{itemize}
    \item \textit{To find the value that is to be used in the outer query}
    \item \textit{To find the rows that are to be used in the outer query}
    \item \textit{To find the columns that are to be used in the outer query}
\end{itemize}

\item \textit{Explain SQL SubQueries with ALL, ANY, EXISTS, SOME, With UPDATE}\\

Sql subqueries can be used with the following operators, ALL, ANY, EXISTS, SOME, IN, NOT IN, =, <>, >, <, >=, <=, !=, IS NULL, IS NOT NULL, BETWEEN, NOT BETWEEN, LIKE, NOT LIKE, etc.\\

They can be explained as follows:

\begin{itemize}
    \item \textit{ALL} - Returns true if the subquery returns a value that is less than or equal to all the values in the subquery.
    \item  \textit{ANY} - Returns true if the subquery returns a value that is less than or equal to any of the values in the subquery.
    \item \textit{EXISTS} - Returns true if the subquery returns any rows.
    \item  \textit{SOME} - Returns true if the subquery returns a value that is less than or equal to any of the values in the subquery.
    \item  \textit{IN} - Returns true if the subquery returns a value that is equal to any of the values in the subquery.
    \item \textit{NOT IN} - Returns true if the subquery returns a value that is not equal to any of the values in the subquery
\end{itemize}

\item \textit{How to get groupwise data from a table. What is use of Having Clause}\\

Groupwise data can be obtained using the GROUP BY clause. The HAVING clause is used to filter the groups.\\

\textit{An Example query would be }

\begin{lstlisting}[language=sql]
SELECT column_name, aggregate_function(column_name) from table_name having aggregate_function(column_name) operator value group by column_name;
\end{lstlisting}

\item \textit{What is ‘having’ clause and when to use it?}\\

The Having clause is used to filter the groups. It is used with the GROUP BY clause.\\

\item \textit{How to display data from View. Are the views updatable? Explain}\\

Data from a view can be displayed using the SELECT statement.\\

Views are not updatable. They are read-only. This is because the view is not a physical table. It is a virtual table.\\
\end{enumerate}



\end{document}