% This is a Basic Assignment Paper but with like Code and stuff allowed in it, there is also url, hyperlinks from contents included. 

\documentclass[11pt]{article}

% Preamble

\usepackage[margin=1in]{geometry}
\usepackage{amsfonts, amsmath, amssymb}
\usepackage{fancyhdr, float, graphicx}
\usepackage[utf8]{inputenc} % Required for inputting international characters
\usepackage[T1]{fontenc} % Output font encoding for international characters
\usepackage{fouriernc} % Use the New Century Schoolbook font
\usepackage[nottoc, notlot, notlof]{tocbibind}
\usepackage{listings}
\usepackage{xcolor}
\usepackage{blindtext}
\usepackage{hyperref}
\hypersetup{
    colorlinks=true,
    linkcolor=black,
    filecolor=magenta,      
    urlcolor=cyan,
    pdfpagemode=FullScreen,
    }

\definecolor{codegreen}{rgb}{0,0.6,0}
\definecolor{codegray}{rgb}{0.5,0.5,0.5}
\definecolor{codepurple}{rgb}{0.58,0,0.82}
\definecolor{backcolour}{rgb}{0.95,0.95,0.92}

\lstdefinestyle{mystyle}{
    backgroundcolor=\color{backcolour},   
    commentstyle=\color{codegreen},
    keywordstyle=\color{magenta},
    numberstyle=\tiny\color{codegray},
    stringstyle=\color{codepurple},
    basicstyle=\ttfamily\footnotesize,
    breakatwhitespace=false,         
    breaklines=true,                 
    captionpos=b,                    
    keepspaces=true,                 
    numbers=left,                    
    numbersep=5pt,                  
    showspaces=false,                
    showstringspaces=false,
    showtabs=false,                  
    tabsize=2
}

\lstset{style=mystyle}

% Header and Footer
\pagestyle{fancy}
\fancyhead{}
\fancyfoot{}
\fancyhead[L]{\textit{\Large{Database Management Systems Assignment 3}}}
%\fancyhead[R]{\textit{something}}
\fancyfoot[C]{\thepage}
\renewcommand{\footrulewidth}{1pt}



% Other Doc Editing
% \parindent 0ex
%\renewcommand{\baselinestretch}{1.5}

\begin{document}

\begin{titlepage}
	\centering

	%---------------------------NAMES-------------------------------

	\huge\textsc{
		MIT World Peace University
	}\\

	\vspace{0.75\baselineskip} % space after Uni Name

	\LARGE{
		Database Management Systems\\
		Second Year B. Tech, Semester 4
	}

	\vfill % space after Sub Name

	%--------------------------TITLE-------------------------------

	\rule{\textwidth}{1.6pt}\vspace*{-\baselineskip}\vspace*{2pt}
	\rule{\textwidth}{0.6pt}
	\vspace{0.75\baselineskip} % Whitespace above the title



	\huge{\textsc{
			Learning SQL DML Commands\\
			\textit{Data Manipulation Language}
		}} \\



	\vspace{0.5\baselineskip} % Whitespace below the title
	\rule{\textwidth}{0.6pt}\vspace*{-\baselineskip}\vspace*{2.8pt}
	\rule{\textwidth}{1.6pt}

	\vspace{1\baselineskip} % Whitespace after the title block

	%--------------------------SUBTITLE --------------------------	

	\LARGE\textsc{
		Assignment No. 3
	} % Subtitle or further description
	\vfill

	%--------------------------AUTHOR-------------------------------

	Prepared By
	\vspace{0.5\baselineskip} % Whitespace before the editors

	\Large{
		Krishnaraj Thadesar \\
		Cyber Security and Forensics\\
		Batch A1, PA 20
	}


	\vspace{0.5\baselineskip} % Whitespace below the editor list
	\today

\end{titlepage}


\tableofcontents
\thispagestyle{empty}
\clearpage

\setcounter{page}{1}

\section{Aim}
Write suitable DML and select command to manipulate and retrieve requested data from tables.
\section{Objectives}
\begin{enumerate}
	\item DML (Insert, Update, Delete) commands,
	\item SQL Select- Logical, IN, Negation, NULL, Comparison Operators.
	\item Where Clause, Between AND, Exists, ALL, LIKE
\end{enumerate}

\section{Problem Statement}

\section{Theory}
\subsection{SQL Data Manipulation Language (DML)}

\subsubsection{What is Data Manipulation Language?}
\begin{quote}
	\textit{Data Manipulation Language (DML) is a computer language used to access and manipulate data stored in a database. It is used to retrieve, insert, update, and delete data in a database. }
\end{quote}
\subsubsection{DML Commands}

The following are the Commands that are used in DML:

\begin{enumerate}
	\item SELECT - Retrieves data from a database.
	\item INSERT - Inserts data into a table.
	\item UPDATE - Updates existing data within a table.
	\item DELETE - Deletes existing data within a table.
\end{enumerate}

\subsection{DML Command Syntax and Examples}

\begin{enumerate}
	\item SELECT - Retrieves data from a database.
	      \begin{verbatim}
		SELECT column1, column2, ...
		FROM table_name;
	\end{verbatim}
	\item INSERT - Inserts data into a table.
	      \begin{verbatim}
		INSERT INTO table_name (column1, column2, column3, ...)
		VALUES (value1, value2, value3, ...);
	\end{verbatim}
	\item UPDATE - Updates existing data within a table.
	      \begin{verbatim}
		UPDATE table_name
		SET column1 = value1, column2 = value2, ...
		WHERE condition;
	\end{verbatim}
	\item DELETE - Deletes existing data within a table.
	      \begin{verbatim}
		DELETE FROM table_name WHERE condition;
	\end{verbatim}
\end{enumerate}

\subsection{SELECT query}
\subsubsection{What is SELECT query?}
\begin{quote}
	\textit{The SELECT statement is used to select data from a database. The data returned is stored in a result table, called the result-set.}
\end{quote}

\subsubsection{SELECT Syntax}

\begin{verbatim}
	SELECT column_name(s)
	FROM table_name
	WHERE column_name operator value;
\end{verbatim}

\subsubsection{SELECT Operators}

The following are the Operators that are used in SELECT:

\begin{enumerate}
	\item \textbf{AND} - Returns rows where both conditions are true.
	\item \textbf{OR} - Returns rows where either condition is true.
	\item \textbf{NOT} - Returns rows where the condition(s) is not true.
	\item \textbf{BETWEEN} - Returns rows where the value is within a range of two values.
	\item \textbf{LIKE} - Returns rows where the value matches a pattern.
	\item \textbf{IN} - Returns rows where the value matches any value in a list.
\end{enumerate}

\subsubsection{Examples of the SELECT Query}




\begin{enumerate}
	\item
	      \begin{verbatim}
	SELECT * FROM CUSTOMERS;
	\end{verbatim}
	\item
	      \begin{verbatim}
	SELECT * FROM CUSTOMERS WHERE CUST_ID = 1;
	\end{verbatim}
	\item
	      \begin{verbatim}
	SELECT * FROM CUSTOMERS WHERE CUST_ID = 1 AND CUST_NAME = 'Krishnaraj';
	\end{verbatim}
	\item
	      \begin{verbatim}
	SELECT * FROM CUSTOMERS WHERE CUST_ID = 1 OR CUST_NAME = 'Krishnaraj';
	\end{verbatim}
	\item
	      \begin{verbatim}
	SELECT * FROM CUSTOMERS WHERE NOT CUST_ID = 1;
	\end{verbatim}
	\item
	      \begin{verbatim}
	SELECT * FROM CUSTOMERS WHERE CUST_ID BETWEEN 1 AND 5;
	\end{verbatim}
	\item
	      \begin{verbatim}
	SELECT * FROM CUSTOMERS WHERE CUST_NAME LIKE 'Krish%';
	\end{verbatim}
	\item
	      \begin{verbatim}
	SELECT * FROM CUSTOMERS WHERE CUST_ID IN (1, 2, 3);
	\end{verbatim}
\end{enumerate}

\subsection{SQL Operators}

\subsubsection{What are SQL Operators?}
\begin{quote}
	\textit{Operators are special symbols in SQL that allow you to perform specific operations on data.}
\end{quote}

\subsubsection{SQL Operators}

The following are the Operators that are used in SQL:

\begin{enumerate}
	\item \textbf{Arithmetic Operators} - Used to perform mathematical operations on numbers.
	\item \textbf{Comparison Operators} - Used to compare values.
	\item \textbf{Logical Operators} - Used to combine two or more conditions.
	\item \textbf{Misc Operators} - Used to perform other operations.
\end{enumerate}

\subsubsection{Arithmetic Operators}

The following are the Arithmetic Operators that are used in SQL:

\begin{enumerate}
	\item \textbf{+} - Addition
	\item \textbf{-} - Subtraction
	\item \textbf{*} - Multiplication
	\item \textbf{/} - Division
	\item \textbf{MOD} - Modulus
\end{enumerate}

\subsubsection{Comparison Operators}

The following are the Comparison Operators that are used in SQL:

\begin{enumerate}
	\item \textbf{=} - Equal
	\item \textbf{<>} - Not equal. Note: In some versions of SQL this operator may be written as !=
	\item \textbf{>} - Greater than
	\item \textbf{<} - Less than
	\item \textbf{>=} - Greater than or equal
	\item \textbf{<=} - Less than or equal
	\item \textbf{BETWEEN} - Between an inclusive range
	\item \textbf{LIKE} - Search for a pattern
	\item \textbf{IN} - To specify multiple possible values for a column
\end{enumerate}

\subsubsection{Logical Operators}

The following are the Logical Operators that are used in SQL:

\begin{enumerate}
	\item \textbf{AND} - Logical AND
	\item \textbf{OR} - Logical OR
	\item \textbf{NOT} - Logical NOT
\end{enumerate}


\section{Platform}
\textbf{Operating System}: Arch Linux x86-64 \\
\textbf{IDEs or Text Editors Used}: Draw.io for Drawing the ER diagram.

% \section{Pseudo Code or Algorithm}

\section{Input}
Given Database from the Problem Statement for the Assignment for our batch. (A1 PA 20)
\section{Output}
\lstinputlisting[language=SQL]{../../Programs/Assignment_3.md}
\section{Conclusion}
Thus, we have learned SQL DML commands, SELECT Command with SQL operators thoroughly.
\clearpage

\section{FAQ}
\begin{enumerate}
	\item \textbf{What is the difference between Truncate table and Drop table command?}

	      \begin{enumerate}
		      \item \textit{Truncate table command deletes all the records from the table and resets the identity column to 1.}
		      \item \textit{Drop table command deletes the table and all the records from the table.}
		      \item \textit{Truncate table command is faster than Drop table command.}
		      \item \textit{Truncate table command cannot be rolled back.}
		      \item \textit{Drop table command can be rolled back.}
	      \end{enumerate}

		  \textbf{Example:}
	      \begin{enumerate}
		      \item \textit{Truncate table command}
		            \begin{verbatim}
			Truncate table CUSTOMERS;
		\end{verbatim}
		      \item \textit{Drop table command}
		            \begin{verbatim}
			Drop table CUSTOMERS;
		\end{verbatim}
	      \end{enumerate}

	\item \textbf{How is the pattern matching done in the SQL?}

	      \begin{enumerate}
		      \item \textit{The pattern matching is done using the LIKE operator.}
		      \item \textit{The pattern matching is done using the wildcard characters.}
		      \item \textit{The wildcard characters are:}
		            \begin{itemize}
			            \item \textit{\% - Represents zero or more characters.}
			            \item \textit{\_ - Represents a single character.}
			            \item \textit{[charlist] - Represents any single character in charlist.}
		            \end{itemize}
	      \end{enumerate}

	      \textbf{The Syntax of the command is:}
	      \begin{verbatim}
			SELECT column_name(s) FROM table_name WHERE column_name LIKE pattern;
		\end{verbatim}

	      \textbf{Example:}
	      \begin{verbatim}
			SELECT * FROM CUSTOMERS WHERE CUST_NAME LIKE 'Emp%';
			SELECT * FROM STUDENTS WHERE CUST_NAME LIKE 'AssignmentNumber_';
		\end{verbatim}


	\item \textbf{Write a DELETE command to delete all the records from CUSTOMERS table.}

	      \begin{verbatim}
			DELETE FROM CUSTOMERS;
		\end{verbatim}
\end{enumerate}

\end{document}