% This is a Basic Assignment Paper but with like Code and stuff allowed in it, there is also url, hyperlinks from contents included. 

\documentclass[11pt]{article}

% Preamble

\usepackage[margin=1in]{geometry}
\usepackage{amsfonts, amsmath, amssymb}
\usepackage{fancyhdr, float, graphicx}
\usepackage[utf8]{inputenc} % Required for inputting international characters
\usepackage[T1]{fontenc} % Output font encoding for international characters
\usepackage{fouriernc} % Use the New Century Schoolbook font
\usepackage[nottoc, notlot, notlof]{tocbibind}
\usepackage{listings}
\usepackage{xcolor}
\usepackage{blindtext}
\usepackage{hyperref}
\hypersetup{
    colorlinks=true,
    linkcolor=black,
    filecolor=magenta,      
    urlcolor=cyan,
    pdfpagemode=FullScreen,
    }

\definecolor{codegreen}{rgb}{0,0.6,0}
\definecolor{codegray}{rgb}{0.5,0.5,0.5}
\definecolor{codepurple}{rgb}{0.58,0,0.82}
\definecolor{backcolour}{rgb}{0.95,0.95,0.92}

\lstdefinestyle{mystyle}{
    backgroundcolor=\color{backcolour},   
    commentstyle=\color{codegreen},
    keywordstyle=\color{magenta},
    numberstyle=\tiny\color{codegray},
    stringstyle=\color{codepurple},
    basicstyle=\ttfamily\footnotesize,
    breakatwhitespace=false,         
    breaklines=true,                 
    captionpos=b,                    
    keepspaces=true,                 
    numbers=left,                    
    numbersep=5pt,                  
    showspaces=false,                
    showstringspaces=false,
    showtabs=false,                  
    tabsize=2
}

\lstset{style=mystyle}

% Header and Footer
\pagestyle{fancy}
\fancyhead{}
\fancyfoot{}
\fancyhead[L]{\textit{\Large{Database Management Systems Assignment 7}}}
%\fancyhead[R]{\textit{something}}
\fancyfoot[C]{\thepage}
\renewcommand{\footrulewidth}{1pt}



% Other Doc Editing
% \parindent 0ex
%\renewcommand{\baselinestretch}{1.5}

\begin{document}

\begin{titlepage}
    \centering

    %---------------------------NAMES-------------------------------

    \huge\textsc{
        MIT World Peace University
    }\\

    \vspace{0.75\baselineskip} % space after Uni Name

    \LARGE{
        Database Management Systems\\
        Second Year B. Tech, Semester 4
    }

    \vfill % space after Sub Name

    %--------------------------TITLE-------------------------------

    \rule{\textwidth}{1.6pt}\vspace*{-\baselineskip}\vspace*{2pt}
    \rule{\textwidth}{0.6pt}
    \vspace{0.75\baselineskip} % Whitespace above the title



    \huge{\textsc{
            Create Triggers using PL/SQL
        }} \\



    \vspace{0.5\baselineskip} % Whitespace below the title
    \rule{\textwidth}{0.6pt}\vspace*{-\baselineskip}\vspace*{2.8pt}
    \rule{\textwidth}{1.6pt}

    \vspace{1\baselineskip} % Whitespace after the title block

    %--------------------------SUBTITLE --------------------------	

    \LARGE\textsc{
        Assignment No. 7
    } % Subtitle or further description
    \vfill

    %--------------------------AUTHOR-------------------------------

    Prepared By
    \vspace{0.5\baselineskip} % Whitespace before the editors

    \Large{
        Krishnaraj Thadesar \\
        Cyber Security and Forensics\\
        Batch A1, PA 20
    }


    \vspace{0.5\baselineskip} % Whitespace below the editor list
    \today

\end{titlepage}


\tableofcontents
\thispagestyle{empty}
\clearpage

\setcounter{page}{1}

\section{Aim}
Write PL/SQL Triggers for the creation of insert trigger, delete trigger and update trigger on the
given problem statements.

\section{Objectives}
\begin{enumerate}
    \item To study and use Triggers using MySQL PL/SQL block
\end{enumerate}


\section{Problem Statement}
Create tables and solve given queries

\section{Theory}

\subsection{Triggers in PL/SQL}
A trigger is a named PL/SQL unit that is stored in the database and can be invoked repeatedly. A trigger automatically executes whenever an event associated with a table occurs. There are two types of triggers based on the which level it is triggered. They are:

\begin{enumerate}
    \item Row Level Trigger
    \item Statement Level Trigger
    \item Database Level Trigger
    \item Instead of Trigger
    \item DDL Trigger
    \item System Trigger
    \item Compound Trigger
\end{enumerate}

\subsection{Advantages of Triggers}

\begin{enumerate}
    \item Triggers can be used to enforce complex business rules.
    \item Triggers can be used to enforce complex integrity constraints.
    \item Triggers can be used to propagate data to other tables.
    \item Triggers can be used to log all changes to a table.
    \item Triggers can be used to prevent invalid transactions.
    \item Triggers can be used to notify external applications of database changes.
\end{enumerate}


\subsection{Disadvantages of Triggers}

\begin{enumerate}
    \item Triggers are hidden within the database and can be difficult to find.
    \item Triggers are not visible in the SQL source code.
    \item Triggers can be difficult to debug.
    \item Triggers can cause performance issues.
    \item Triggers can cause infinite loops.
    \item Triggers can cause locking issues.
    \item Triggers can cause cascading failures.
    \item Triggers can cause unpredictable results.
    \item Triggers can cause security issues.
\end{enumerate}

\subsection{Usage of Triggers}
Usages of Triggers

\begin{enumerate}
    \item Auditing
    \item Data Integrity
    \item Data Validation
    \item Notification
    \item Replication
    \item Security
    \item Synchronization
\end{enumerate}

\subsection{Difference between Stored Procedure and Trigger}
\begin{enumerate}
    \item Triggers are invoked automatically in response to the associated DML statement.
    \item Stored procedures are invoked explicitly by users or applications.
    \item Triggers are attached to tables and are implicitly invoked.
    \item Stored procedures are stand-alone and are explicitly invoked.
    \item Triggers execute implicitly in response to DML statements on the table with which they are associated.
    \item Stored procedures execute explicitly when they are invoked by a user or application.
    \item Triggers execute under the security privileges of the owner of the trigger.
    \item Stored procedures execute under the security privileges of the user who invokes them.
    \item Triggers are attached to tables and are implicitly invoked.
    \item Stored procedures are stand-alone and are explicitly invoked.
    \item Triggers execute implicitly in response to DML statements on the table with which they are associated.
    \item Stored procedures execute explicitly when they are invoked by a user or application.
    \item Triggers execute under the security privileges of the owner of the trigger.
    \item Stored procedures execute under the security privileges of the user who invokes them.
\end{enumerate}

Syntax of Triggers: 

\begin{lstlisting}[language=sql]
    CREATE TRIGGER Trigger_Name  
    [ BEFORE | AFTER ]  [ Insert | Update | Delete]  
    ON [Table_Name]  
    [ FOR EACH ROW | FOR EACH COLUMN ]  
    AS  
    Set of SQL Statement  
\end{lstlisting}

\subsection{Types of Triggers}
\begin{enumerate}
    \item Row Level Trigger
          \begin{itemize}
              \item Row level trigger is triggered when a row is inserted, updated or deleted from a table.
              \item Row level trigger is declared at row level.
              \item Row level trigger is used to perform an action when a DML event (INSERT, UPDATE, DELETE) occurs.
              \item Row level trigger can be used to enforce complex business rules or integrity constraints.
          \end{itemize}

    \item Statement Level Trigger
          \begin{itemize}
              \item Statement level trigger is triggered when a DML event (INSERT, UPDATE, DELETE) occurs.
              \item Statement level trigger is declared at statement level.
              \item Statement level trigger is used to perform an action when a DML event (INSERT, UPDATE, DELETE) occurs.
              \item Statement level trigger can be used to enforce complex business rules or integrity constraints.
          \end{itemize}

    \item Database Level Trigger
          \begin{itemize}
              \item Database level trigger is triggered when a DDL event (CREATE, ALTER, DROP, RENAME, TRUNCATE) occurs.
              \item Database level trigger is declared at database level.
              \item Database level trigger is used to perform an action when a DDL event (CREATE, ALTER, DROP, RENAME, TRUNCATE) occurs.
              \item Database level trigger can be used to enforce complex business rules or integrity constraints.
          \end{itemize}

    \item Instead of Trigger
          \begin{itemize}
              \item Instead of trigger is triggered when a DML event (INSERT, UPDATE, DELETE) occurs.
              \item Instead of trigger is declared at view level.
              \item Instead of trigger is used to perform an action when a DML event (INSERT, UPDATE, DELETE) occurs.
              \item Instead of trigger can be used to enforce complex business rules or integrity constraints.
          \end{itemize}

    \item DDL Trigger
          \begin{itemize}
              \item DDL trigger is triggered when a DDL event (CREATE, ALTER, DROP, RENAME, TRUNCATE) occurs.
              \item DDL trigger is declared at database level.
              \item DDL trigger is used to perform an action when a DDL event (CREATE, ALTER,
    \end{itemize}
\end{enumerate}

\subsection{Trigger Level}

Trigger Levels are used to specify when the trigger should be fired. There are two types of trigger levels. They are:
\begin{enumerate}
    \item Before Trigger
          \begin{itemize}
              \item Before trigger is triggered before the triggering statement is executed.
              \item Before trigger is declared using BEFORE keyword.
              \item Before trigger is used to perform an action before the triggering statement is executed.
              \item Before trigger can be used to enforce complex business rules or integrity constraints.
          \end{itemize}

    \item After Trigger
          \begin{itemize}
              \item After trigger is triggered after the triggering statement is executed.
              \item After trigger is declared using AFTER keyword.
              \item After trigger is used to perform an action after the triggering statement is executed.
              \item After trigger can be used to enforce complex business rules or integrity constraints.
          \end{itemize}
        
\end{enumerate}

\subsection{Trigger Events}

Trigger Events are used to specify when the trigger should be fired. There are three types of trigger events. They are:

\begin{enumerate}
    \item Insert Trigger
          \begin{itemize}
              \item Insert trigger is triggered when an insert event occurs.
              \item Insert trigger is declared using INSERT keyword.
              \item Insert trigger is used to perform an action when an insert event occurs.
              \item Insert trigger can be used to enforce complex business rules or integrity constraints.
          \end{itemize}

    \item Update Trigger
          \begin{itemize}
              \item Update trigger is triggered when an update event occurs.
              \item Update trigger is declared using UPDATE keyword.
              \item Update trigger is used to perform an action when an update event occurs.
              \item Update trigger can be used to enforce complex business rules or integrity constraints.
          \end{itemize}

    \item Delete Trigger
          \begin{itemize}
              \item Delete trigger is triggered when a delete event occurs.
              \item Delete trigger is declared using DELETE keyword.
              \item Delete trigger is used to perform an action when a delete event occurs.
              \item Delete trigger can be used to enforce complex business rules or integrity constraints.
          \end{itemize}

\end{enumerate}

An SQL example for trigger events is given below:
\begin{lstlisting}[language=sql]
CREATE TRIGGER trigger_name
    BEFORE INSERT ON table_name
    FOR EACH ROW
    BEGIN
        -- trigger body
    END;
\end{lstlisting}

\subsection{NEW and OLD Clause /Trigger Variables}

Clauses are used to specify the columns of the table. There are two types of clauses. They are:

\begin{enumerate}
    \item NEW Clause
          \begin{itemize}
              \item NEW clause is used to specify the columns of the table that are affected by the triggering statement.
              \item NEW clause is used in INSERT and UPDATE triggers.
              \item NEW clause is used to specify the columns of the table that are affected by the triggering statement.
              \item NEW clause is used in INSERT and UPDATE triggers.
          \end{itemize}

    \item OLD Clause
          \begin{itemize}
              \item OLD clause is used to specify the columns of the table that are affected by the triggering statement.
              \item OLD clause is used in UPDATE and DELETE triggers.
              \item OLD clause is used to specify the columns of the table that are affected by the triggering statement.
              \item OLD clause is used in UPDATE and DELETE triggers.
          \end{itemize}
\end{enumerate}

An SQL Example for NEW and OLD clause would be:
\begin{lstlisting}[language=sql]
    CREATE TRIGGER trigger_name
    BEFORE INSERT ON table_name
    FOR EACH ROW
    BEGIN
        -- trigger body
        IF NEW.column_name = 'value' THEN
            -- trigger body
        END IF;
    END;
\end{lstlisting}

\subsection{Dropping Triggers}

Dropping Triggers is used to drop the trigger from the database. An SQL Example for this would be: 

\begin{lstlisting}[language=sql]
    DROP TRIGGER trigger_name;
\end{lstlisting}

\section{Platform}
\textbf{Operating System}: Arch Linux x86-64 \\
\textbf{IDEs or Text Editors Used}: Draw.io for Drawing the ER diagram. \\

\section{Input}
Given Database from the Problem Statement for the Assignment for our batch. (A1 PA 20)

\section{Queries}
\lstinputlisting[language=SQL]{../../Programs/Assignment_7_queries.sql}
\section{Outputs}
\lstinputlisting[language=SQL]{../../Programs/Assignment_7.md}

\section{Conclusion}
Thus, we have learned creating and using triggers in SQL. We have also learned the advantages and disadvantages of using triggers in SQL.
\clearpage

\section{FAQ}
\begin{enumerate}
    \item \textbf{Enlist Advantages of Triggers ?}\\

        Advantages of Triggers:
          \begin{itemize}
              \item Data Integrity: Triggers can help to ensure that data in a database remains consistent and accurate, by automatically enforcing data validation rules.
              \item Automation: Triggers can automate repetitive or complex database operations, such as updating related records or sending notifications based on specific events.
              \item Security: Triggers can be used to implement security policies, such as preventing unauthorized access or monitoring user activity.
              \item Performance: Triggers can improve database performance by reducing the number of queries needed to perform certain operations.
              \item Audit trail: Triggers can be used to create an audit trail of changes made to data, which can be useful for compliance or troubleshooting purposes.
          \end{itemize}

    \item \textbf{Enlist Disadvantages of Triggers ?}\\

        Disadvantages of Triggers:
          \begin{itemize}
              \item Complexity: Triggers can add complexity to database design and maintenance, making it harder to understand and modify the database schema.

              \item Debugging: Debugging triggers can be difficult, as they are executed automatically and can be triggered by a wide range of events.

              \item Performance: Triggers can also have a negative impact on database performance, particularly if they are poorly designed or executed frequently.

              \item Scalability: Triggers can make it more difficult to scale a database system, as they can increase the number of transactions and the amount of data being processed.
          \end{itemize}
    \item \textbf{What are the Applications of Triggers.}\\

        Applications of Triggers:
          \begin{itemize}
              \item Data validation: Triggers can be used to enforce data validation rules, such as checking that certain fields are not left blank or that certain values fall within a specified range.

              \item Data synchronization: Triggers can be used to synchronize data between different tables or databases, ensuring that related records are always up-to-date.

              \item Notifications: Triggers can be used to send notifications or alerts based on specific events, such as when a new record is added or an existing record is updated.

              \item Security: Triggers can be used to implement security policies, such as preventing unauthorized access or monitoring user activity.

              \item Audit trail: Triggers can be used to create an audit trail of changes made to data, which can be useful for compliance or troubleshooting purposes.
          \end{itemize}
\end{enumerate}
\end{document}