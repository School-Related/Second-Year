% This is a Basic Assignment Paper but with like Code and stuff allowed in it, there is also url, hyperlinks from contents included. 

\documentclass[11pt]{article}

% Preamble

\usepackage[margin=1in]{geometry}
\usepackage{amsfonts, amsmath, amssymb}
\usepackage{fancyhdr, float, graphicx}
\usepackage[utf8]{inputenc} % Required for inputting international characters
\usepackage[T1]{fontenc} % Output font encoding for international characters
\usepackage{fouriernc} % Use the New Century Schoolbook font
\usepackage[nottoc, notlot, notlof]{tocbibind}
\usepackage{listings}
\usepackage{xcolor}
\usepackage{blindtext}
\usepackage{hyperref}
\hypersetup{
    colorlinks=true,
    linkcolor=black,
    filecolor=magenta,      
    urlcolor=cyan,
    pdfpagemode=FullScreen,
    }

\definecolor{codegreen}{rgb}{0,0.6,0}
\definecolor{codegray}{rgb}{0.5,0.5,0.5}
\definecolor{codepurple}{rgb}{0.58,0,0.82}
\definecolor{backcolour}{rgb}{0.95,0.95,0.92}

\lstdefinestyle{mystyle}{
    backgroundcolor=\color{backcolour},   
    commentstyle=\color{codegreen},
    keywordstyle=\color{magenta},
    numberstyle=\tiny\color{codegray},
    stringstyle=\color{codepurple},
    basicstyle=\ttfamily\footnotesize,
    breakatwhitespace=false,         
    breaklines=true,                 
    captionpos=b,                    
    keepspaces=true,                 
    numbers=left,                    
    numbersep=5pt,                  
    showspaces=false,                
    showstringspaces=false,
    showtabs=false,                  
    tabsize=2
}

\lstset{style=mystyle}

% Header and Footer
\pagestyle{fancy}
\fancyhead{}
\fancyfoot{}
\fancyhead[L]{\textit{\Large{Information and Cycbersecurity - 2nd Year B. Tech}}}
%\fancyhead[R]{\textit{something}}
\fancyfoot[C]{\thepage}
\renewcommand{\footrulewidth}{1pt}



% Other Doc Editing
% \parindent 0ex
%\renewcommand{\baselinestretch}{1.5}

\begin{document}

\begin{titlepage}
	\centering

	%---------------------------NAMES-------------------------------

	\huge\textsc{
		MIT World Peace University
	}\\

	\vspace{0.75\baselineskip} % space after Uni Name

	\LARGE{
		Information and Cybersecurity\\
		Second Year B. Tech, Semester 1
	}

	\vfill % space after Sub Name

	%--------------------------TITLE-------------------------------

	\rule{\textwidth}{1.6pt}\vspace*{-\baselineskip}\vspace*{2pt}
	\rule{\textwidth}{0.6pt}
	\vspace{0.75\baselineskip} % Whitespace above the title



	\huge{\textsc{
			Classical Cryptographic Techniques - \\
			\textit{"Fiestal Cipher"}
		}} \\



	\vspace{0.5\baselineskip} % Whitespace below the title
	\rule{\textwidth}{0.6pt}\vspace*{-\baselineskip}\vspace*{2.8pt}
	\rule{\textwidth}{1.6pt}

	\vspace{1\baselineskip} % Whitespace after the title block

	%--------------------------SUBTITLE --------------------------	

	\LARGE\textsc{
		Lab Assignment 2
	} % Subtitle or further description
	\vfill

	%--------------------------AUTHOR-------------------------------

	Prepared By
	\vspace{0.5\baselineskip} % Whitespace before the editors

	\Large{
		Krishnaraj Thadesar \\
		Cyber Security and Forensics\\
		Batch A1, PA 20
	}


	\vspace{0.5\baselineskip} % Whitespace below the editor list
	\today

\end{titlepage}


\tableofcontents
\thispagestyle{empty}
\clearpage

\setcounter{page}{1}

\section{Aim}
Write a program using JAVA or Python or C++ to implement Feistal Cipher structure

\section{Objectives}
To understand the concepts of symmetric key cryptographic system.

\section{Theory}
\subsection{Symmetric Key Cryptography}

Symmetric key cryptography is a cryptographic system in which the same key is used for both encryption and decryption. The key is shared between the sender and the receiver. The sender encrypts the message using the key and sends it to the receiver. The receiver decrypts the message using the same key. The key is kept secret and is never sent along with the message.

The most commonly used symmetric key algorithm is the Data Encryption Standard (DES). It uses a 64-bit block size and a 56-bit key. The 64-bit block is divided into two halves of 32-bits each. The key is also divided into two halves of 28-bits each. The first half of the key is used to generate 16 subkeys. Each subkey is 48-bits long. The first 28-bits of the key are shifted left by 1 bit. The first 28-bits of the key are then shifted left by 1 bit. The second 28-bits of the key are shifted left by 1 bit. The second 28-bits of the key are shifted left by 1 bit. This process is repeated for the remaining 16 rounds. The 16 subkeys are then used to encrypt the message.

\subsection{Feistal Cipher}

The Feistal cipher is a symmetric key cryptographic system. It is a block cipher that uses a symmetric key. The key is shared between the sender and the receiver. The sender encrypts the message using the key and sends it to the receiver. The receiver decrypts the message using the same key. The key is kept secret and is never sent along with the message.

The Feistal cipher uses a 64-bit block size and a 56-bit key. The 64-bit block is divided into two halves of 32-bits each. The key is also divided into two halves of 28-bits each. The first half of the key is used to generate 16 subkeys. Each subkey is 48-bits long. The first 28-bits of the key are shifted left by 1 bit. The first 28-bits of the key are then shifted left by 1 bit. The second 28-bits of the key are shifted left by 1 bit. The second 28-bits of the key are shifted left by 1 bit. This process is repeated for the remaining 16 rounds. The 16 subkeys are then used to encrypt the message.

\section{Platform}
\textbf{Operating System}: Arch Linux x86-64 \\
\textbf{IDEs or Text Editors Used}: Visual Studio Code\\
\textbf{Compilers or Interpreters} : Python 3.10.1\\

\section{Input and Output}
\begin{verbatim}
    The plain text, key
    [1, 1, 1, 1, 0, 0, 1, 1] [1, 0, 1, 0, 0, 0, 0, 0, 1, 0]
    The left and right keys are :  [1, 0, 1, 0, 0, 1, 0, 0] [0, 1, 0, 0, 0, 0, 1, 1]
    Starting to cipher. 
    The cipher text is :  [0, 1, 0, 0, 0, 0, 0, 1]
\end{verbatim}
\section{Code}
\lstinputlisting[language=Python, caption="Fiestal Cipher"]{../Programs/Assignment_2/fiestel_cipher.py}

\section{Conclusion}
Thus, learnt about the different kinds of ciphers, classical cryptographic techniques, and how to implement some of them in python.
\clearpage

\section{FAQ}

\begin{enumerate}
	\item \textbf{Differentiate between stream and block ciphers.}\\
	      Answer:
	      \begin{enumerate}
		      \item Stream ciphers encrypt the data one bit at a time. Block ciphers encrypt the data in blocks of fixed size.
		      \item Stream ciphers are faster than block ciphers.
		      \item Block ciphers are more secure than stream ciphers.
		      \item Stream ciphers are more suitable for real-time applications.
		      \item Block ciphers are more suitable for bulk data encryption.
		      \item Stream ciphers are more suitable for applications where the data is encrypted and decrypted in a single pass.
		      \item Block ciphers are more suitable for applications where the data is encrypted and decrypted in multiple passes.
	      \end{enumerate}

	\item \textbf{Write advantages and disadvantages of DES algorithm.}\\
	      Answer:
	      Advantages:
	      \begin{enumerate}
		      \item It is a fast, simple, efficient, and secure algorithm.
		      \item The algorithm has been in use since 1977. Technically, no weaknesses have been found in the algorithm. Brute force attacks are still the most efficient attacks against the DES algorithm.
		      \item DES is the standard set by the US Government. The government recertifies DES every five years, and has to ask for its replacement if the need arises.
		      \item The American National Standards Institute (ANSI) and International Organization for Standardization (ISO) have declared DES as a standard as well. This means that the algorithm is open to the public—to learn and implement.
		      \item DES was designed for hardware; it is fast in hardware, but only relatively fast in software.
	      \end{enumerate}

	      Disadvantages:
	      \begin{enumerate}
		      \item Probably the biggest disadvantage of the DES algorithm is the key size of 56-bit. There are chips available that can encrypt and decrypt a million DES operations in a second. A DES cracking machine that can search all the keys in about seven hours is available for 1 million.
		      \item DES can be implemented quickly on hardware. But since it was not designed for software, it is relatively slow on it.
		      \item It has become easier to break the encrypted code in DES as the technology is steadily improving. Nowadays, AES is preferred over DES.
		      \item DES uses a single key for encryption as well as decryption as it is a type of symmetric encryption technique. In case that one key is lost, we will not be able to receive decipherable data at all.
	      \end{enumerate}

	\item \textbf{Explain block cipher modes of operations.}\\
	      Answer:
	      \begin{enumerate}
		      \item Electronic Code Book (ECB) - In this mode, the plaintext is divided into blocks of equal size. Each block is encrypted separately. The same plaintext block will always result in the same ciphertext block. This mode is not secure as it is vulnerable to a known plaintext attack.
		      \item Cipher Block Chaining (CBC) - In this mode, the plaintext is divided into blocks of equal size. Each block is XORed with the previous ciphertext block before being encrypted. This mode is more secure than ECB as it is not vulnerable to a known plaintext attack.
		      \item Cipher Feedback (CFB)
		      \item Output Feedback (OFB)
		      \item Counter (CTR) 
	      \end{enumerate}
\end{enumerate}

\end{document}