% This is a Basic Assignment Paper but with like Code and stuff allowed in it, there is also url, hyperlinks from contents included. 

\documentclass[11pt]{article}

% Preamble

\usepackage[margin=1in]{geometry}
\usepackage{amsfonts, amsmath, amssymb}
\usepackage{fancyhdr, float, graphicx}
\usepackage[utf8]{inputenc} % Required for inputting international characters
\usepackage[T1]{fontenc} % Output font encoding for international characters
\usepackage{fouriernc} % Use the New Century Schoolbook font
\usepackage[nottoc, notlot, notlof]{tocbibind}
\usepackage{listings}
\usepackage{xcolor}
\usepackage{blindtext}
\usepackage{hyperref}


\definecolor{codegreen}{rgb}{0,0.6,0}
\definecolor{codegray}{rgb}{0.5,0.5,0.5}
\definecolor{codepurple}{rgb}{0.58,0,0.82}
\definecolor{backcolour}{rgb}{0.95,0.95,0.92}

\lstdefinestyle{mystyle}{
    backgroundcolor=\color{backcolour},   
    commentstyle=\color{codegreen},
    keywordstyle=\color{magenta},
    numberstyle=\tiny\color{codegray},
    stringstyle=\color{codepurple},
    basicstyle=\ttfamily\footnotesize,
    breakatwhitespace=false,         
    breaklines=true,                 
    captionpos=b,                    
    keepspaces=true,                 
    numbers=left,                    
    numbersep=5pt,                  
    showspaces=false,                
    showstringspaces=false,
    showtabs=false,                  
    tabsize=2
}

\lstset{style=mystyle}

% Header and Footer
\pagestyle{fancy}
\fancyhead{}
\fancyfoot{}
\fancyhead[L]{\textit{\Large{Python Programming Assignment 2}}}
%\fancyhead[R]{\textit{something}}
\fancyfoot[C]{\thepage}
\renewcommand{\footrulewidth}{1pt}

\usepackage[breakable]{tcolorbox}
\usepackage{parskip} % Stop auto-indenting (to mimic markdown behaviour)


% Basic figure setup, for now with no caption control since it's done
% automatically by Pandoc (which extracts ![](path) syntax from Markdown).
\usepackage{graphicx}
% Maintain compatibility with old templates. Remove in nbconvert 6.0
\let\Oldincludegraphics\includegraphics
% Ensure that by default, figures have no caption (until we provide a
% proper Figure object with a Caption API and a way to capture that
% in the conversion process - todo).
\usepackage{caption}
\DeclareCaptionFormat{nocaption}{}
\captionsetup{format=nocaption,aboveskip=0pt,belowskip=0pt}

\usepackage{float}
\floatplacement{figure}{H} % forces figures to be placed at the correct location
\usepackage{xcolor} % Allow colors to be defined
\usepackage{enumerate} % Needed for markdown enumerations to work
\usepackage{geometry} % Used to adjust the document margins
\usepackage{amsmath} % Equations
\usepackage{amssymb} % Equations
\usepackage{textcomp} % defines textquotesingle
% Hack from http://tex.stackexchange.com/a/47451/13684:
\AtBeginDocument{%
	\def\PYZsq{\textquotesingle}% Upright quotes in Pygmentized code
}
\usepackage{upquote} % Upright quotes for verbatim code
\usepackage{eurosym} % defines \euro

\usepackage{iftex}
% \ifPDFTeX
% 	\usepackage[T1]{fontenc}
% 	\IfFileExists{alphabeta.sty}{
% 		  \usepackage{alphabeta}
% 	  }{
% 		  \usepackage[mathletters]{ucs}
% 		  \usepackage[utf8x]{inputenc}
% 	  }
% \else
% 	\usepackage{fontspec}
% 	\usepackage{unicode-math}
% \fi

\usepackage{fancyvrb} % verbatim replacement that allows latex
\usepackage{grffile} % extends the file name processing of package graphics
					 % to support a larger range
\makeatletter % fix for old versions of grffile with XeLaTeX
\@ifpackagelater{grffile}{2019/11/01}
{
  % Do nothing on new versions
}
{
  \def\Gread@@xetex#1{%
	\IfFileExists{"\Gin@base".bb}%
	{\Gread@eps{\Gin@base.bb}}%
	{\Gread@@xetex@aux#1}%
  }
}
\makeatother
\usepackage[Export]{adjustbox} % Used to constrain images to a maximum size
\adjustboxset{max size={0.9\linewidth}{0.9\paperheight}}

% The hyperref package gives us a pdf with properly built
% internal navigation ('pdf bookmarks' for the table of contents,
% internal cross-reference links, web links for URLs, etc.)
\usepackage{hyperref}
% The default LaTeX title has an obnoxious amount of whitespace. By default,
% titling removes some of it. It also provides customization options.
\usepackage{titling}
\usepackage{longtable} % longtable support required by pandoc >1.10
\usepackage{booktabs}  % table support for pandoc > 1.12.2
\usepackage{array}     % table support for pandoc >= 2.11.3
\usepackage{calc}      % table minipage width calculation for pandoc >= 2.11.1
\usepackage[inline]{enumitem} % IRkernel/repr support (it uses the enumerate* environment)
\usepackage[normalem]{ulem} % ulem is needed to support strikethroughs (\sout)
							% normalem makes italics be italics, not underlines
\usepackage{mathrsfs}



% Colors for the hyperref package
\definecolor{urlcolor}{rgb}{0,.145,.698}
\definecolor{linkcolor}{rgb}{.71,0.21,0.01}
\definecolor{citecolor}{rgb}{.12,.54,.11}

% ANSI colors
\definecolor{ansi-black}{HTML}{3E424D}
\definecolor{ansi-black-intense}{HTML}{282C36}
\definecolor{ansi-red}{HTML}{E75C58}
\definecolor{ansi-red-intense}{HTML}{B22B31}
\definecolor{ansi-green}{HTML}{00A250}
\definecolor{ansi-green-intense}{HTML}{007427}
\definecolor{ansi-yellow}{HTML}{DDB62B}
\definecolor{ansi-yellow-intense}{HTML}{B27D12}
\definecolor{ansi-blue}{HTML}{208FFB}
\definecolor{ansi-blue-intense}{HTML}{0065CA}
\definecolor{ansi-magenta}{HTML}{D160C4}
\definecolor{ansi-magenta-intense}{HTML}{A03196}
\definecolor{ansi-cyan}{HTML}{60C6C8}
\definecolor{ansi-cyan-intense}{HTML}{258F8F}
\definecolor{ansi-white}{HTML}{C5C1B4}
\definecolor{ansi-white-intense}{HTML}{A1A6B2}
\definecolor{ansi-default-inverse-fg}{HTML}{FFFFFF}
\definecolor{ansi-default-inverse-bg}{HTML}{000000}

% common color for the border for error outputs.
\definecolor{outerrorbackground}{HTML}{FFDFDF}

% commands and environments needed by pandoc snippets
% extracted from the output of `pandoc -s`
\providecommand{\tightlist}{%
  \setlength{\itemsep}{0pt}\setlength{\parskip}{0pt}}
\DefineVerbatimEnvironment{Highlighting}{Verbatim}{commandchars=\\\{\}}
% Add ',fontsize=\small' for more characters per line
\newenvironment{Shaded}{}{}
\newcommand{\KeywordTok}[1]{\textcolor[rgb]{0.00,0.44,0.13}{\textbf{{#1}}}}
\newcommand{\DataTypeTok}[1]{\textcolor[rgb]{0.56,0.13,0.00}{{#1}}}
\newcommand{\DecValTok}[1]{\textcolor[rgb]{0.25,0.63,0.44}{{#1}}}
\newcommand{\BaseNTok}[1]{\textcolor[rgb]{0.25,0.63,0.44}{{#1}}}
\newcommand{\FloatTok}[1]{\textcolor[rgb]{0.25,0.63,0.44}{{#1}}}
\newcommand{\CharTok}[1]{\textcolor[rgb]{0.25,0.44,0.63}{{#1}}}
\newcommand{\StringTok}[1]{\textcolor[rgb]{0.25,0.44,0.63}{{#1}}}
\newcommand{\CommentTok}[1]{\textcolor[rgb]{0.38,0.63,0.69}{\textit{{#1}}}}
\newcommand{\OtherTok}[1]{\textcolor[rgb]{0.00,0.44,0.13}{{#1}}}
\newcommand{\AlertTok}[1]{\textcolor[rgb]{1.00,0.00,0.00}{\textbf{{#1}}}}
\newcommand{\FunctionTok}[1]{\textcolor[rgb]{0.02,0.16,0.49}{{#1}}}
\newcommand{\RegionMarkerTok}[1]{{#1}}
\newcommand{\ErrorTok}[1]{\textcolor[rgb]{1.00,0.00,0.00}{\textbf{{#1}}}}
\newcommand{\NormalTok}[1]{{#1}}

% Additional commands for more recent versions of Pandoc
\newcommand{\ConstantTok}[1]{\textcolor[rgb]{0.53,0.00,0.00}{{#1}}}
\newcommand{\SpecialCharTok}[1]{\textcolor[rgb]{0.25,0.44,0.63}{{#1}}}
\newcommand{\VerbatimStringTok}[1]{\textcolor[rgb]{0.25,0.44,0.63}{{#1}}}
\newcommand{\SpecialStringTok}[1]{\textcolor[rgb]{0.73,0.40,0.53}{{#1}}}
\newcommand{\ImportTok}[1]{{#1}}
\newcommand{\DocumentationTok}[1]{\textcolor[rgb]{0.73,0.13,0.13}{\textit{{#1}}}}
\newcommand{\AnnotationTok}[1]{\textcolor[rgb]{0.38,0.63,0.69}{\textbf{\textit{{#1}}}}}
\newcommand{\CommentVarTok}[1]{\textcolor[rgb]{0.38,0.63,0.69}{\textbf{\textit{{#1}}}}}
\newcommand{\VariableTok}[1]{\textcolor[rgb]{0.10,0.09,0.49}{{#1}}}
\newcommand{\ControlFlowTok}[1]{\textcolor[rgb]{0.00,0.44,0.13}{\textbf{{#1}}}}
\newcommand{\OperatorTok}[1]{\textcolor[rgb]{0.40,0.40,0.40}{{#1}}}
\newcommand{\BuiltInTok}[1]{{#1}}
\newcommand{\ExtensionTok}[1]{{#1}}
\newcommand{\PreprocessorTok}[1]{\textcolor[rgb]{0.74,0.48,0.00}{{#1}}}
\newcommand{\AttributeTok}[1]{\textcolor[rgb]{0.49,0.56,0.16}{{#1}}}
\newcommand{\InformationTok}[1]{\textcolor[rgb]{0.38,0.63,0.69}{\textbf{\textit{{#1}}}}}
\newcommand{\WarningTok}[1]{\textcolor[rgb]{0.38,0.63,0.69}{\textbf{\textit{{#1}}}}}


% Define a nice break command that doesn't care if a line doesn't already
% exist.
\def\br{\hspace*{\fill} \\* }
% Math Jax compatibility definitions
\def\gt{>}
\def\lt{<}
\let\Oldtex\TeX
\let\Oldlatex\LaTeX
\renewcommand{\TeX}{\textrm{\Oldtex}}
\renewcommand{\LaTeX}{\textrm{\Oldlatex}}
% Document parameters
% Document title
\title{Assignment\_1}





% Pygments definitions
\makeatletter
\def\PY@reset{\let\PY@it=\relax \let\PY@bf=\relax%
\let\PY@ul=\relax \let\PY@tc=\relax%
\let\PY@bc=\relax \let\PY@ff=\relax}
\def\PY@tok#1{\csname PY@tok@#1\endcsname}
\def\PY@toks#1+{\ifx\relax#1\empty\else%
\PY@tok{#1}\expandafter\PY@toks\fi}
\def\PY@do#1{\PY@bc{\PY@tc{\PY@ul{%
\PY@it{\PY@bf{\PY@ff{#1}}}}}}}
\def\PY#1#2{\PY@reset\PY@toks#1+\relax+\PY@do{#2}}

\@namedef{PY@tok@w}{\def\PY@tc##1{\textcolor[rgb]{0.73,0.73,0.73}{##1}}}
\@namedef{PY@tok@c}{\let\PY@it=\textit\def\PY@tc##1{\textcolor[rgb]{0.24,0.48,0.48}{##1}}}
\@namedef{PY@tok@cp}{\def\PY@tc##1{\textcolor[rgb]{0.61,0.40,0.00}{##1}}}
\@namedef{PY@tok@k}{\let\PY@bf=\textbf\def\PY@tc##1{\textcolor[rgb]{0.00,0.50,0.00}{##1}}}
\@namedef{PY@tok@kp}{\def\PY@tc##1{\textcolor[rgb]{0.00,0.50,0.00}{##1}}}
\@namedef{PY@tok@kt}{\def\PY@tc##1{\textcolor[rgb]{0.69,0.00,0.25}{##1}}}
\@namedef{PY@tok@o}{\def\PY@tc##1{\textcolor[rgb]{0.40,0.40,0.40}{##1}}}
\@namedef{PY@tok@ow}{\let\PY@bf=\textbf\def\PY@tc##1{\textcolor[rgb]{0.67,0.13,1.00}{##1}}}
\@namedef{PY@tok@nb}{\def\PY@tc##1{\textcolor[rgb]{0.00,0.50,0.00}{##1}}}
\@namedef{PY@tok@nf}{\def\PY@tc##1{\textcolor[rgb]{0.00,0.00,1.00}{##1}}}
\@namedef{PY@tok@nc}{\let\PY@bf=\textbf\def\PY@tc##1{\textcolor[rgb]{0.00,0.00,1.00}{##1}}}
\@namedef{PY@tok@nn}{\let\PY@bf=\textbf\def\PY@tc##1{\textcolor[rgb]{0.00,0.00,1.00}{##1}}}
\@namedef{PY@tok@ne}{\let\PY@bf=\textbf\def\PY@tc##1{\textcolor[rgb]{0.80,0.25,0.22}{##1}}}
\@namedef{PY@tok@nv}{\def\PY@tc##1{\textcolor[rgb]{0.10,0.09,0.49}{##1}}}
\@namedef{PY@tok@no}{\def\PY@tc##1{\textcolor[rgb]{0.53,0.00,0.00}{##1}}}
\@namedef{PY@tok@nl}{\def\PY@tc##1{\textcolor[rgb]{0.46,0.46,0.00}{##1}}}
\@namedef{PY@tok@ni}{\let\PY@bf=\textbf\def\PY@tc##1{\textcolor[rgb]{0.44,0.44,0.44}{##1}}}
\@namedef{PY@tok@na}{\def\PY@tc##1{\textcolor[rgb]{0.41,0.47,0.13}{##1}}}
\@namedef{PY@tok@nt}{\let\PY@bf=\textbf\def\PY@tc##1{\textcolor[rgb]{0.00,0.50,0.00}{##1}}}
\@namedef{PY@tok@nd}{\def\PY@tc##1{\textcolor[rgb]{0.67,0.13,1.00}{##1}}}
\@namedef{PY@tok@s}{\def\PY@tc##1{\textcolor[rgb]{0.73,0.13,0.13}{##1}}}
\@namedef{PY@tok@sd}{\let\PY@it=\textit\def\PY@tc##1{\textcolor[rgb]{0.73,0.13,0.13}{##1}}}
\@namedef{PY@tok@si}{\let\PY@bf=\textbf\def\PY@tc##1{\textcolor[rgb]{0.64,0.35,0.47}{##1}}}
\@namedef{PY@tok@se}{\let\PY@bf=\textbf\def\PY@tc##1{\textcolor[rgb]{0.67,0.36,0.12}{##1}}}
\@namedef{PY@tok@sr}{\def\PY@tc##1{\textcolor[rgb]{0.64,0.35,0.47}{##1}}}
\@namedef{PY@tok@ss}{\def\PY@tc##1{\textcolor[rgb]{0.10,0.09,0.49}{##1}}}
\@namedef{PY@tok@sx}{\def\PY@tc##1{\textcolor[rgb]{0.00,0.50,0.00}{##1}}}
\@namedef{PY@tok@m}{\def\PY@tc##1{\textcolor[rgb]{0.40,0.40,0.40}{##1}}}
\@namedef{PY@tok@gh}{\let\PY@bf=\textbf\def\PY@tc##1{\textcolor[rgb]{0.00,0.00,0.50}{##1}}}
\@namedef{PY@tok@gu}{\let\PY@bf=\textbf\def\PY@tc##1{\textcolor[rgb]{0.50,0.00,0.50}{##1}}}
\@namedef{PY@tok@gd}{\def\PY@tc##1{\textcolor[rgb]{0.63,0.00,0.00}{##1}}}
\@namedef{PY@tok@gi}{\def\PY@tc##1{\textcolor[rgb]{0.00,0.52,0.00}{##1}}}
\@namedef{PY@tok@gr}{\def\PY@tc##1{\textcolor[rgb]{0.89,0.00,0.00}{##1}}}
\@namedef{PY@tok@ge}{\let\PY@it=\textit}
\@namedef{PY@tok@gs}{\let\PY@bf=\textbf}
\@namedef{PY@tok@gp}{\let\PY@bf=\textbf\def\PY@tc##1{\textcolor[rgb]{0.00,0.00,0.50}{##1}}}
\@namedef{PY@tok@go}{\def\PY@tc##1{\textcolor[rgb]{0.44,0.44,0.44}{##1}}}
\@namedef{PY@tok@gt}{\def\PY@tc##1{\textcolor[rgb]{0.00,0.27,0.87}{##1}}}
\@namedef{PY@tok@err}{\def\PY@bc##1{{\setlength{\fboxsep}{\string -\fboxrule}\fcolorbox[rgb]{1.00,0.00,0.00}{1,1,1}{\strut ##1}}}}
\@namedef{PY@tok@kc}{\let\PY@bf=\textbf\def\PY@tc##1{\textcolor[rgb]{0.00,0.50,0.00}{##1}}}
\@namedef{PY@tok@kd}{\let\PY@bf=\textbf\def\PY@tc##1{\textcolor[rgb]{0.00,0.50,0.00}{##1}}}
\@namedef{PY@tok@kn}{\let\PY@bf=\textbf\def\PY@tc##1{\textcolor[rgb]{0.00,0.50,0.00}{##1}}}
\@namedef{PY@tok@kr}{\let\PY@bf=\textbf\def\PY@tc##1{\textcolor[rgb]{0.00,0.50,0.00}{##1}}}
\@namedef{PY@tok@bp}{\def\PY@tc##1{\textcolor[rgb]{0.00,0.50,0.00}{##1}}}
\@namedef{PY@tok@fm}{\def\PY@tc##1{\textcolor[rgb]{0.00,0.00,1.00}{##1}}}
\@namedef{PY@tok@vc}{\def\PY@tc##1{\textcolor[rgb]{0.10,0.09,0.49}{##1}}}
\@namedef{PY@tok@vg}{\def\PY@tc##1{\textcolor[rgb]{0.10,0.09,0.49}{##1}}}
\@namedef{PY@tok@vi}{\def\PY@tc##1{\textcolor[rgb]{0.10,0.09,0.49}{##1}}}
\@namedef{PY@tok@vm}{\def\PY@tc##1{\textcolor[rgb]{0.10,0.09,0.49}{##1}}}
\@namedef{PY@tok@sa}{\def\PY@tc##1{\textcolor[rgb]{0.73,0.13,0.13}{##1}}}
\@namedef{PY@tok@sb}{\def\PY@tc##1{\textcolor[rgb]{0.73,0.13,0.13}{##1}}}
\@namedef{PY@tok@sc}{\def\PY@tc##1{\textcolor[rgb]{0.73,0.13,0.13}{##1}}}
\@namedef{PY@tok@dl}{\def\PY@tc##1{\textcolor[rgb]{0.73,0.13,0.13}{##1}}}
\@namedef{PY@tok@s2}{\def\PY@tc##1{\textcolor[rgb]{0.73,0.13,0.13}{##1}}}
\@namedef{PY@tok@sh}{\def\PY@tc##1{\textcolor[rgb]{0.73,0.13,0.13}{##1}}}
\@namedef{PY@tok@s1}{\def\PY@tc##1{\textcolor[rgb]{0.73,0.13,0.13}{##1}}}
\@namedef{PY@tok@mb}{\def\PY@tc##1{\textcolor[rgb]{0.40,0.40,0.40}{##1}}}
\@namedef{PY@tok@mf}{\def\PY@tc##1{\textcolor[rgb]{0.40,0.40,0.40}{##1}}}
\@namedef{PY@tok@mh}{\def\PY@tc##1{\textcolor[rgb]{0.40,0.40,0.40}{##1}}}
\@namedef{PY@tok@mi}{\def\PY@tc##1{\textcolor[rgb]{0.40,0.40,0.40}{##1}}}
\@namedef{PY@tok@il}{\def\PY@tc##1{\textcolor[rgb]{0.40,0.40,0.40}{##1}}}
\@namedef{PY@tok@mo}{\def\PY@tc##1{\textcolor[rgb]{0.40,0.40,0.40}{##1}}}
\@namedef{PY@tok@ch}{\let\PY@it=\textit\def\PY@tc##1{\textcolor[rgb]{0.24,0.48,0.48}{##1}}}
\@namedef{PY@tok@cm}{\let\PY@it=\textit\def\PY@tc##1{\textcolor[rgb]{0.24,0.48,0.48}{##1}}}
\@namedef{PY@tok@cpf}{\let\PY@it=\textit\def\PY@tc##1{\textcolor[rgb]{0.24,0.48,0.48}{##1}}}
\@namedef{PY@tok@c1}{\let\PY@it=\textit\def\PY@tc##1{\textcolor[rgb]{0.24,0.48,0.48}{##1}}}
\@namedef{PY@tok@cs}{\let\PY@it=\textit\def\PY@tc##1{\textcolor[rgb]{0.24,0.48,0.48}{##1}}}

\def\PYZbs{\char`\\}
\def\PYZus{\char`\_}
\def\PYZob{\char`\{}
\def\PYZcb{\char`\}}
\def\PYZca{\char`\^}
\def\PYZam{\char`\&}
\def\PYZlt{\char`\<}
\def\PYZgt{\char`\>}
\def\PYZsh{\char`\#}
\def\PYZpc{\char`\%}
\def\PYZdl{\char`\$}
\def\PYZhy{\char`\-}
\def\PYZsq{\char`\'}
\def\PYZdq{\char`\"}
\def\PYZti{\char`\~}
% for compatibility with earlier versions
\def\PYZat{@}
\def\PYZlb{[}
\def\PYZrb{]}
\makeatother


% For linebreaks inside Verbatim environment from package fancyvrb.
\makeatletter
	\newbox\Wrappedcontinuationbox
	\newbox\Wrappedvisiblespacebox
	\newcommand*\Wrappedvisiblespace {\textcolor{red}{\textvisiblespace}}
	\newcommand*\Wrappedcontinuationsymbol {\textcolor{red}{\llap{\tiny$\m@th\hookrightarrow$}}}
	\newcommand*\Wrappedcontinuationindent {3ex }
	\newcommand*\Wrappedafterbreak {\kern\Wrappedcontinuationindent\copy\Wrappedcontinuationbox}
	% Take advantage of the already applied Pygments mark-up to insert
	% potential linebreaks for TeX processing.
	%        {, <, #, %, $, ' and ": go to next line.
	%        _, }, ^, &, >, - and ~: stay at end of broken line.
	% Use of \textquotesingle for straight quote.
	\newcommand*\Wrappedbreaksatspecials {%
		\def\PYGZus{\discretionary{\char`\_}{\Wrappedafterbreak}{\char`\_}}%
		\def\PYGZob{\discretionary{}{\Wrappedafterbreak\char`\{}{\char`\{}}%
		\def\PYGZcb{\discretionary{\char`\}}{\Wrappedafterbreak}{\char`\}}}%
		\def\PYGZca{\discretionary{\char`\^}{\Wrappedafterbreak}{\char`\^}}%
		\def\PYGZam{\discretionary{\char`\&}{\Wrappedafterbreak}{\char`\&}}%
		\def\PYGZlt{\discretionary{}{\Wrappedafterbreak\char`\<}{\char`\<}}%
		\def\PYGZgt{\discretionary{\char`\>}{\Wrappedafterbreak}{\char`\>}}%
		\def\PYGZsh{\discretionary{}{\Wrappedafterbreak\char`\#}{\char`\#}}%
		\def\PYGZpc{\discretionary{}{\Wrappedafterbreak\char`\%}{\char`\%}}%
		\def\PYGZdl{\discretionary{}{\Wrappedafterbreak\char`\$}{\char`\$}}%
		\def\PYGZhy{\discretionary{\char`\-}{\Wrappedafterbreak}{\char`\-}}%
		\def\PYGZsq{\discretionary{}{\Wrappedafterbreak\textquotesingle}{\textquotesingle}}%
		\def\PYGZdq{\discretionary{}{\Wrappedafterbreak\char`\"}{\char`\"}}%
		\def\PYGZti{\discretionary{\char`\~}{\Wrappedafterbreak}{\char`\~}}%
	}
	% Some characters . , ; ? ! / are not pygmentized.
	% This macro makes them "active" and they will insert potential linebreaks
	\newcommand*\Wrappedbreaksatpunct {%
		\lccode`\~`\.\lowercase{\def~}{\discretionary{\hbox{\char`\.}}{\Wrappedafterbreak}{\hbox{\char`\.}}}%
		\lccode`\~`\,\lowercase{\def~}{\discretionary{\hbox{\char`\,}}{\Wrappedafterbreak}{\hbox{\char`\,}}}%
		\lccode`\~`\;\lowercase{\def~}{\discretionary{\hbox{\char`\;}}{\Wrappedafterbreak}{\hbox{\char`\;}}}%
		\lccode`\~`\:\lowercase{\def~}{\discretionary{\hbox{\char`\:}}{\Wrappedafterbreak}{\hbox{\char`\:}}}%
		\lccode`\~`\?\lowercase{\def~}{\discretionary{\hbox{\char`\?}}{\Wrappedafterbreak}{\hbox{\char`\?}}}%
		\lccode`\~`\!\lowercase{\def~}{\discretionary{\hbox{\char`\!}}{\Wrappedafterbreak}{\hbox{\char`\!}}}%
		\lccode`\~`\/\lowercase{\def~}{\discretionary{\hbox{\char`\/}}{\Wrappedafterbreak}{\hbox{\char`\/}}}%
		\catcode`\.\active
		\catcode`\,\active
		\catcode`\;\active
		\catcode`\:\active
		\catcode`\?\active
		\catcode`\!\active
		\catcode`\/\active
		\lccode`\~`\~
	}
\makeatother

\let\OriginalVerbatim=\Verbatim
\makeatletter
\renewcommand{\Verbatim}[1][1]{%
	%\parskip\z@skip
	\sbox\Wrappedcontinuationbox {\Wrappedcontinuationsymbol}%
	\sbox\Wrappedvisiblespacebox {\FV@SetupFont\Wrappedvisiblespace}%
	\def\FancyVerbFormatLine ##1{\hsize\linewidth
		\vtop{\raggedright\hyphenpenalty\z@\exhyphenpenalty\z@
			\doublehyphendemerits\z@\finalhyphendemerits\z@
			\strut ##1\strut}%
	}%
	% If the linebreak is at a space, the latter will be displayed as visible
	% space at end of first line, and a continuation symbol starts next line.
	% Stretch/shrink are however usually zero for typewriter font.
	\def\FV@Space {%
		\nobreak\hskip\z@ plus\fontdimen3\font minus\fontdimen4\font
		\discretionary{\copy\Wrappedvisiblespacebox}{\Wrappedafterbreak}
		{\kern\fontdimen2\font}%
	}%

	% Allow breaks at special characters using \PYG... macros.
	\Wrappedbreaksatspecials
	% Breaks at punctuation characters . , ; ? ! and / need catcode=\active
	\OriginalVerbatim[#1,codes*=\Wrappedbreaksatpunct]%
}
\makeatother

% Exact colors from NB
\definecolor{incolor}{HTML}{303F9F}
\definecolor{outcolor}{HTML}{D84315}
\definecolor{cellborder}{HTML}{CFCFCF}
\definecolor{cellbackground}{HTML}{F7F7F7}

% prompt
\makeatletter
\newcommand{\boxspacing}{\kern\kvtcb@left@rule\kern\kvtcb@boxsep}
\makeatother
\newcommand{\prompt}[4]{
	{\ttfamily\llap{{\color{#2}[#3]:\hspace{3pt}#4}}\vspace{-\baselineskip}}
}



% Prevent overflowing lines due to hard-to-break entities
\sloppy
% Setup hyperref package
\hypersetup{
  breaklinks=true,  % so long urls are correctly broken across lines
  colorlinks=true,
  urlcolor=urlcolor,
  linkcolor=linkcolor,
  citecolor=citecolor,
  }
% Slightly bigger margins than the latex defaults

\geometry{verbose,tmargin=1in,bmargin=1in,lmargin=1in,rmargin=1in}
\hypersetup{
    colorlinks=true,
    linkcolor=black,
    filecolor=magenta,      
    urlcolor=cyan,
    pdfpagemode=FullScreen,
    }



% Other Doc Editing
% \parindent 0ex
%\renewcommand{\baselinestretch}{1.5}

\begin{document}

\begin{titlepage}
	\centering

	%---------------------------NAMES-------------------------------

	\huge\textsc{
		MIT World Peace University
	}\\

	\vspace{0.75\baselineskip} % space after Uni Name

	\LARGE{
		Python Programming\\
		Second Year B. Tech, Semester 4
	}

	\vfill % space after Sub Name

	%--------------------------TITLE-------------------------------

	\rule{\textwidth}{1.6pt}\vspace*{-\baselineskip}\vspace*{2pt}
	\rule{\textwidth}{0.6pt}
	\vspace{0.75\baselineskip} % Whitespace above the title



	\huge{\textsc{
			Different Operations on the Dictionary and Tuple Data Structures
		}} \\



	\vspace{0.5\baselineskip} % Whitespace below the title
	\rule{\textwidth}{0.6pt}\vspace*{-\baselineskip}\vspace*{2.8pt}
	\rule{\textwidth}{1.6pt}

	\vspace{1\baselineskip} % Whitespace after the title block

	%--------------------------SUBTITLE --------------------------	

	\LARGE\textsc{
		Assignment No. 5
	} % Subtitle or further description
	\vfill

	%--------------------------AUTHOR-------------------------------

	Prepared By
	\vspace{0.5\baselineskip} % Whitespace before the editors

	\Large{
		Krishnaraj Thadesar \\
		Cyber Security and Forensics\\
		Batch A1, PA 20
	}


	\vspace{0.5\baselineskip} % Whitespace below the editor list
	\today

\end{titlepage}

\tableofcontents
\thispagestyle{empty}
\clearpage

\setcounter{page}{1}

\section{Aim}
Write a python program to create, append and remove etc. operation on Dictionary and
Tuple.
\section{Objectives}
\begin{enumerate}
	\item To learn and implement Dictionary and Tuple Data Structure.
\end{enumerate}

\section{Theory}
\subsection{Different Operations performed on Dictionaries}
Following are the different operations performed on Dictionaries:

\begin{itemize}
	\item \textbf{\textit{Creating a Dictionary}}
	\item \textbf{\textit{Accessing elements from a Dictionary}}
	\item \textbf{\textit{Changing and Adding Dictionary elements}}
	\item \textbf{\textit{Removing elements from a Dictionary}}
\end{itemize}

\subsubsection{Creating a Dictionary}
A dictionary in Python is an unordered collection of data values, used to store data values like a map, which unlike other Data Types that hold only single value as an element, Dictionary holds key:value pair. Key-value is provided in the dictionary to make it more optimized. Each key-value pair in a Dictionary is separated by a colon :, whereas each key is separated by a ‘comma’.
\begin{lstlisting}[language=Python]
	>>> d={1:'a',2:'b'}
	>>> d
	{1: 'a', 2: 'b'}
\end{lstlisting}

\subsubsection{Accessing elements from a Dictionary}
While indexing is used with other container types to access values, a Dictionary uses keys. Key can be used either inside square brackets [] or with the get() method. If we use the square brackets [], then a KeyError is raised in case a key is not found in the dictionary. On the other hand, the get() method returns None if the key is not found.
\begin{lstlisting}[language=Python]
	>>> d={1:'a',2:'b'}
	>>> d[1]
	'a'
	>>> d.get(1)
	'a'
\end{lstlisting}

\subsubsection{Changing and Adding Dictionary elements}
In Python, Dictionary are mutable. It means that we can change the content of a dictionary any time. To add a new item to the dictionary, we can use the familiar square brackets along with the new key.
\begin{lstlisting}[language=Python]
	>>> d={1:'a',2:'b'}
	>>> d[3]='c'
	>>> d
	{1: 'a', 2: 'b', 3: 'c'}
\end{lstlisting}

\subsubsection{Removing elements from a Dictionary}
There are various methods to remove items from a dictionary:
\begin{itemize}
	\item \textbf{\textit{Using pop() method}}
	\item \textbf{\textit{Using popitem() method}}
	\item \textbf{\textit{Using del keyword}}
	\item \textbf{\textit{Using clear() method}}
\end{itemize}

\subsection{Different Operations performed on Tuples}
Following are the different operations performed on Tuples:

\begin{itemize}
	\item \textbf{\textit{Creating a Tuple}}
	\item \textbf{\textit{Accessing elements from a Tuple}}
	\item \textbf{\textit{Changing and Adding Tuple elements}}
	\item \textbf{\textit{Removing elements from a Tuple}}
\end{itemize}

\subsubsection{Creating a Tuple}
A tuple is a collection which is ordered and unchangeable. In Python tuples are written with round brackets.
\begin{lstlisting}[language=Python]
	>>> t = ("apple", "banana", "cherry")
	>>> t
	('apple', 'banana', 'cherry')
\end{lstlisting}

\subsubsection{Accessing elements from a Tuple}
You can access tuple items by referring to the index number, inside square brackets.
\begin{lstlisting}[language=Python]
	>>> t = ("apple", "banana", "cherry")
	>>> t[1]
	'banana'
\end{lstlisting}

\subsubsection{Changing and Adding Tuple elements}
Tuples are unchangeable, so you cannot add items to it after it has been created.
\begin{lstlisting}[language=Python]
	>>> t = ("apple", "banana", "cherry")
	>>> t[3] = "orange" 
	Traceback (most recent call last):
	  File "<stdin>", line 1, in <module>
	TypeError: 'tuple' object does not support item assignment
\end{lstlisting}

\subsubsection{Removing elements from a Tuple}
Tuples are unchangeable, so you cannot remove items from it, but you can delete the tuple completely.
\begin{lstlisting}[language=Python]
	>>> t = ("apple", "banana", "cherry")
	>>> del t
\end{lstlisting}

\section{Input and Output}
\subsection{Input}
Different Dictionary and Tuple Data Structure and different operations
\subsection{Output}
Display Different operation performed on Dictionary and Tuple Data Structure

\hypertarget{Code}{%
\section{Code}\label{tuples}}

    \hypertarget{write-a-python-program-to-create-append-and-remove-etc.-operation-on-dictionary-and-tuple.}{%
\subsubsection{Write a python program to create, append and remove etc.
operation on Dictionary and
Tuple.}\label{write-a-python-program-to-create-append-and-remove-etc.-operation-on-dictionary-and-tuple.}}

    \hypertarget{creating-a-tuple}{%
\subsubsection{Creating a tuple}\label{creating-a-tuple}}

    \begin{tcolorbox}[breakable, size=fbox, boxrule=1pt, pad at break*=1mm,colback=cellbackground, colframe=cellborder]
\prompt{In}{incolor}{3}{\boxspacing}
\begin{Verbatim}[commandchars=\\\{\}]
\PY{n}{my\PYZus{}tuple} \PY{o}{=} \PY{p}{(}\PY{l+m+mi}{1}\PY{p}{,} \PY{l+m+mi}{2}\PY{p}{,} \PY{l+m+mi}{3}\PY{p}{,} \PY{l+m+mi}{4}\PY{p}{,} \PY{l+m+mi}{4}\PY{p}{,} \PY{l+m+mi}{4}\PY{p}{,} \PY{l+m+mi}{5}\PY{p}{,} \PY{l+m+mi}{6}\PY{p}{,} \PY{l+m+mi}{7}\PY{p}{,} \PY{l+m+mi}{8}\PY{p}{,} \PY{l+m+mi}{9}\PY{p}{,} \PY{l+m+mi}{10}\PY{p}{)}
\PY{n+nb}{print}\PY{p}{(}\PY{n}{my\PYZus{}tuple}\PY{p}{)}
\PY{n+nb}{print}\PY{p}{(}\PY{l+s+s2}{\PYZdq{}}\PY{l+s+s2}{Getting an element from a tuple: }\PY{l+s+s2}{\PYZdq{}}\PY{p}{,} \PY{n}{my\PYZus{}tuple}\PY{p}{[}\PY{l+m+mi}{3}\PY{p}{]}\PY{p}{)}
\PY{n+nb}{print}\PY{p}{(}\PY{l+s+s2}{\PYZdq{}}\PY{l+s+s2}{Index of the first occurrence of 5 is: }\PY{l+s+s2}{\PYZdq{}}\PY{p}{,} \PY{n}{my\PYZus{}tuple}\PY{o}{.}\PY{n}{index}\PY{p}{(}\PY{l+m+mi}{5}\PY{p}{)}\PY{p}{)}
\end{Verbatim}
\end{tcolorbox}

    \hypertarget{appending-to-a-tuple}{%
\subsubsection{Appending to a tuple}\label{appending-to-a-tuple}}

    \begin{tcolorbox}[breakable, size=fbox, boxrule=1pt, pad at break*=1mm,colback=cellbackground, colframe=cellborder]
\prompt{In}{incolor}{29}{\boxspacing}
\begin{Verbatim}[commandchars=\\\{\}]
\PY{c+c1}{\PYZsh{} my\PYZus{}tuple.append(4) \PYZsh{} this wont work as tuples are immutable}
\PY{n}{my\PYZus{}tuple} \PY{o}{=} \PY{n}{my\PYZus{}tuple} \PY{o}{+} \PY{p}{(}\PY{l+m+mi}{1}\PY{p}{,} \PY{l+s+s2}{\PYZdq{}}\PY{l+s+s2}{added element}\PY{l+s+s2}{\PYZdq{}}\PY{p}{)}
\PY{n}{my\PYZus{}tuple}
\end{Verbatim}
\end{tcolorbox}

            \begin{tcolorbox}[breakable, size=fbox, boxrule=.5pt, pad at break*=1mm, opacityfill=0]
\prompt{Out}{outcolor}{29}{\boxspacing}
\begin{Verbatim}[commandchars=\\\{\}]
(1, 2, 3, 4, 5, 6, 7, 8, 9, 10, 1, 1, 'added element', 1, 'added element')
\end{Verbatim}
\end{tcolorbox}
        
    \hypertarget{difference-between-tuple-and-list}{%
\subsubsection{Difference between tuple and
list}\label{difference-between-tuple-and-list}}

    \begin{tcolorbox}[breakable, size=fbox, boxrule=1pt, pad at break*=1mm,colback=cellbackground, colframe=cellborder]
\prompt{In}{incolor}{52}{\boxspacing}
\begin{Verbatim}[commandchars=\\\{\}]
\PY{c+c1}{\PYZsh{} lists are mutable}
\PY{n}{my\PYZus{}list} \PY{o}{=} \PY{p}{[}\PY{l+m+mi}{1}\PY{p}{,} \PY{l+m+mi}{2}\PY{p}{,} \PY{l+m+mi}{3}\PY{p}{,} \PY{l+m+mi}{4}\PY{p}{,} \PY{l+m+mi}{5}\PY{p}{]}
\PY{n}{my\PYZus{}tuple} \PY{o}{=} \PY{n+nb}{tuple}\PY{p}{(}\PY{n}{my\PYZus{}list}\PY{p}{)}

\PY{c+c1}{\PYZsh{} Deleting is allowed even tho you cant remove single elements. }
\PY{k}{del}\PY{p}{(}\PY{n}{my\PYZus{}list}\PY{p}{)}
\PY{c+c1}{\PYZsh{} tuples are not mutable}
\PY{c+c1}{\PYZsh{} my\PYZus{}tuple[0] = 2 \PYZsh{} not allowed}

\PY{c+c1}{\PYZsh{} trying to sort a list}
\PY{n}{my\PYZus{}list} \PY{o}{=} \PY{p}{[}\PY{l+m+mi}{1}\PY{p}{,} \PY{l+m+mi}{2}\PY{p}{,} \PY{l+m+mi}{3}\PY{p}{,} \PY{l+m+mi}{4}\PY{p}{,} \PY{l+m+mi}{5}\PY{p}{,} \PY{l+m+mi}{6}\PY{p}{,} \PY{l+m+mi}{7}\PY{p}{,} \PY{l+m+mi}{8}\PY{p}{,} \PY{l+m+mi}{9}\PY{p}{,} \PY{l+m+mi}{10}\PY{p}{]}
\PY{n+nb}{print}\PY{p}{(}\PY{l+s+s2}{\PYZdq{}}\PY{l+s+s2}{Before sorting: }\PY{l+s+s2}{\PYZdq{}}\PY{p}{,} \PY{n}{my\PYZus{}list}\PY{p}{)}
\PY{n+nb}{print}\PY{p}{(}\PY{l+s+s2}{\PYZdq{}}\PY{l+s+s2}{After sorting: }\PY{l+s+s2}{\PYZdq{}}\PY{p}{,} \PY{n+nb}{sorted}\PY{p}{(}\PY{n}{my\PYZus{}list}\PY{p}{)}\PY{p}{)}

\PY{k}{try}\PY{p}{:} 
    \PY{n}{my\PYZus{}list} \PY{o}{=} \PY{p}{[}\PY{l+m+mi}{1}\PY{p}{,} \PY{l+m+mi}{2}\PY{p}{,} \PY{l+m+mi}{3}\PY{p}{,} \PY{l+m+mi}{4}\PY{p}{,} \PY{l+m+mi}{5}\PY{p}{,} \PY{l+m+mi}{6}\PY{p}{,} \PY{l+m+mi}{7}\PY{p}{,} \PY{l+m+mi}{8}\PY{p}{,} \PY{l+m+mi}{9}\PY{p}{,} \PY{l+m+mi}{10}\PY{p}{,} \PY{l+s+s1}{\PYZsq{}}\PY{l+s+s1}{hi}\PY{l+s+s1}{\PYZsq{}}\PY{p}{]}
    \PY{n+nb}{print}\PY{p}{(}\PY{l+s+s2}{\PYZdq{}}\PY{l+s+s2}{Before sorting: }\PY{l+s+s2}{\PYZdq{}}\PY{p}{,} \PY{n}{my\PYZus{}list}\PY{p}{)}
    \PY{n+nb}{print}\PY{p}{(}\PY{l+s+s2}{\PYZdq{}}\PY{l+s+s2}{After sorting: }\PY{l+s+s2}{\PYZdq{}}\PY{p}{,} \PY{n+nb}{sorted}\PY{p}{(}\PY{n}{my\PYZus{}list}\PY{p}{)}\PY{p}{)}
\PY{k}{except} \PY{n+ne}{TypeError} \PY{k}{as} \PY{n}{e}\PY{p}{:}
    \PY{n+nb}{print}\PY{p}{(}\PY{l+s+s2}{\PYZdq{}}\PY{l+s+s2}{You need all elements of the same type to sort a list. }\PY{l+s+s2}{\PYZdq{}}\PY{p}{)}
    
\PY{c+c1}{\PYZsh{} same goes with tuples}
\end{Verbatim}
\end{tcolorbox}

    \begin{Verbatim}[commandchars=\\\{\}]
Before sorting:  [1, 2, 3, 4, 5, 6, 7, 8, 9, 10]
After sorting:  [1, 2, 3, 4, 5, 6, 7, 8, 9, 10]
Before sorting:  [1, 2, 3, 4, 5, 6, 7, 8, 9, 10, 'hi']
You need all elements of the same type to sort a list.
    \end{Verbatim}

    \hypertarget{some-properties-of-tuple}{%
\subsubsection{Some Properties of
tuple}\label{some-properties-of-tuple}}

    \begin{tcolorbox}[breakable, size=fbox, boxrule=1pt, pad at break*=1mm,colback=cellbackground, colframe=cellborder]
\prompt{In}{incolor}{10}{\boxspacing}
\begin{Verbatim}[commandchars=\\\{\}]
\PY{n}{weird\PYZus{}tuple} \PY{o}{=} \PY{p}{(}\PY{l+m+mi}{4}\PY{p}{)}
\PY{n+nb}{print}\PY{p}{(}\PY{n+nb}{type}\PY{p}{(}\PY{n}{weird\PYZus{}tuple}\PY{p}{)}\PY{p}{)}

\PY{c+c1}{\PYZsh{} You need comma to make it a tuple}
\PY{n}{weird\PYZus{}tuple} \PY{o}{=} \PY{p}{(}\PY{l+m+mi}{4}\PY{p}{,}\PY{p}{)}
\PY{n+nb}{print}\PY{p}{(}\PY{n+nb}{type}\PY{p}{(}\PY{n}{weird\PYZus{}tuple}\PY{p}{)}\PY{p}{)}

\PY{n}{weird\PYZus{}tuple} \PY{o}{=} \PY{p}{(}\PY{p}{(}\PY{l+m+mi}{4}\PY{p}{,}\PY{p}{)}\PY{p}{,}\PY{p}{)}
\PY{n+nb}{print}\PY{p}{(}\PY{n+nb}{type}\PY{p}{(}\PY{n}{weird\PYZus{}tuple}\PY{p}{)}\PY{p}{)}
\end{Verbatim}
\end{tcolorbox}

    \begin{Verbatim}[commandchars=\\\{\}]
<class 'int'>
<class 'tuple'>
<class 'tuple'>
    \end{Verbatim}

    \hypertarget{you-can-change-mutable-elements-inside-a-tuple}{%
\subsubsection{You can change mutable elements inside a
tuple}\label{you-can-change-mutable-elements-inside-a-tuple}}

    \begin{tcolorbox}[breakable, size=fbox, boxrule=1pt, pad at break*=1mm,colback=cellbackground, colframe=cellborder]
\prompt{In}{incolor}{19}{\boxspacing}
\begin{Verbatim}[commandchars=\\\{\}]
\PY{n}{weird\PYZus{}tuple} \PY{o}{=} \PY{p}{(}\PY{l+m+mi}{4}\PY{p}{,} \PY{l+m+mi}{5}\PY{p}{,} \PY{l+m+mi}{6}\PY{p}{,} \PY{l+m+mi}{7}\PY{p}{,} \PY{p}{[}\PY{l+m+mi}{1}\PY{p}{,} \PY{l+m+mi}{2}\PY{p}{,} \PY{l+m+mi}{3}\PY{p}{]}\PY{p}{,} \PY{l+s+s2}{\PYZdq{}}\PY{l+s+s2}{hello}\PY{l+s+s2}{\PYZdq{}}\PY{p}{)}
\PY{n}{weird\PYZus{}tuple}\PY{p}{[}\PY{l+m+mi}{4}\PY{p}{]}\PY{p}{[}\PY{l+m+mi}{1}\PY{p}{]} \PY{o}{=} \PY{l+m+mi}{10} \PY{c+c1}{\PYZsh{} allowed}
\PY{k}{try}\PY{p}{:} 
    \PY{n}{weird\PYZus{}tuple}\PY{p}{[}\PY{l+m+mi}{4}\PY{p}{]} \PY{o}{=} \PY{l+m+mi}{10} \PY{c+c1}{\PYZsh{} not allowed}
\PY{k}{except} \PY{n+ne}{TypeError} \PY{k}{as} \PY{n}{e}\PY{p}{:}
    \PY{n+nb}{print}\PY{p}{(}\PY{l+s+s2}{\PYZdq{}}\PY{l+s+s2}{Thats not allowed}\PY{l+s+s2}{\PYZdq{}}\PY{p}{)}
\PY{k}{try}\PY{p}{:} 
    \PY{n}{weird\PYZus{}tuple}\PY{p}{[}\PY{l+m+mi}{5}\PY{p}{]}\PY{p}{[}\PY{l+m+mi}{1}\PY{p}{]} \PY{o}{=} \PY{l+s+s1}{\PYZsq{}}\PY{l+s+s1}{3}\PY{l+s+s1}{\PYZsq{}} \PY{c+c1}{\PYZsh{} not allowed}
\PY{k}{except} \PY{n+ne}{TypeError} \PY{k}{as} \PY{n}{e}\PY{p}{:}
    \PY{n+nb}{print}\PY{p}{(}\PY{l+s+s2}{\PYZdq{}}\PY{l+s+s2}{thats also not possible as strings are still immutable}\PY{l+s+s2}{\PYZdq{}}\PY{p}{)}
\end{Verbatim}
\end{tcolorbox}

    \begin{Verbatim}[commandchars=\\\{\}]
Thats not allowed
thats also not possible as strings are still immutable
    \end{Verbatim}

    \hypertarget{printing-everything-about-the-tuple}{%
\subsubsection{Printing everything about the
tuple}\label{printing-everything-about-the-tuple}}

    \begin{tcolorbox}[breakable, size=fbox, boxrule=1pt, pad at break*=1mm,colback=cellbackground, colframe=cellborder]
\prompt{In}{incolor}{23}{\boxspacing}
\begin{Verbatim}[commandchars=\\\{\}]
\PY{k}{for} \PY{n}{i} \PY{o+ow}{in} \PY{n}{weird\PYZus{}tuple}\PY{p}{:}
    \PY{n+nb}{print}\PY{p}{(}\PY{n}{i}\PY{p}{)}
\end{Verbatim}
\end{tcolorbox}

    \begin{Verbatim}[commandchars=\\\{\}]
4
5
6
7
[1, 10, 3]
hello
    \end{Verbatim}

    \hypertarget{some-functions-on-tuples}{%
\subsubsection{Some functions on
tuples}\label{some-functions-on-tuples}}

    \begin{tcolorbox}[breakable, size=fbox, boxrule=1pt, pad at break*=1mm,colback=cellbackground, colframe=cellborder]
\prompt{In}{incolor}{80}{\boxspacing}
\begin{Verbatim}[commandchars=\\\{\}]
\PY{n+nb}{print}\PY{p}{(}\PY{n}{my\PYZus{}tuple}\PY{o}{.}\PY{n}{count}\PY{p}{(}\PY{l+m+mi}{4}\PY{p}{)}\PY{p}{)}
\PY{n+nb}{print}\PY{p}{(}\PY{n}{my\PYZus{}tuple}\PY{o}{.}\PY{n}{index}\PY{p}{(}\PY{l+m+mi}{4}\PY{p}{)}\PY{p}{)}
\end{Verbatim}
\end{tcolorbox}

    \begin{Verbatim}[commandchars=\\\{\}]
1
3
    \end{Verbatim}

    \hypertarget{dictionaries}{%
\subsection{Dictionaries}\label{dictionaries}}

    \begin{tcolorbox}[breakable, size=fbox, boxrule=1pt, pad at break*=1mm,colback=cellbackground, colframe=cellborder]
\prompt{In}{incolor}{96}{\boxspacing}
\begin{Verbatim}[commandchars=\\\{\}]
\PY{n}{example\PYZus{}dictionary} \PY{o}{=} \PY{p}{\PYZob{}}
    \PY{l+s+s1}{\PYZsq{}}\PY{l+s+s1}{a}\PY{l+s+s1}{\PYZsq{}} \PY{p}{:} \PY{l+m+mi}{1}\PY{p}{,}
    \PY{l+s+s1}{\PYZsq{}}\PY{l+s+s1}{b}\PY{l+s+s1}{\PYZsq{}} \PY{p}{:} \PY{l+m+mi}{2}\PY{p}{,}
    \PY{l+s+s1}{\PYZsq{}}\PY{l+s+s1}{c}\PY{l+s+s1}{\PYZsq{}} \PY{p}{:} \PY{l+m+mi}{3}\PY{p}{,}
    \PY{l+s+s1}{\PYZsq{}}\PY{l+s+s1}{d}\PY{l+s+s1}{\PYZsq{}} \PY{p}{:} \PY{l+m+mi}{4}\PY{p}{,}
    \PY{l+s+s1}{\PYZsq{}}\PY{l+s+s1}{e}\PY{l+s+s1}{\PYZsq{}} \PY{p}{:} \PY{l+m+mi}{5}\PY{p}{,}
\PY{p}{\PYZcb{}}

\PY{n}{example\PYZus{}dictionary}
\end{Verbatim}
\end{tcolorbox}

            \begin{tcolorbox}[breakable, size=fbox, boxrule=.5pt, pad at break*=1mm, opacityfill=0]
\prompt{Out}{outcolor}{96}{\boxspacing}
\begin{Verbatim}[commandchars=\\\{\}]
\{'a': 1, 'b': 2, 'c': 3, 'd': 4, 'e': 5\}
\end{Verbatim}
\end{tcolorbox}
        
    \hypertarget{different-ways-to-create-a-dictionary}{%
\subsubsection{Different ways to create a
dictionary}\label{different-ways-to-create-a-dictionary}}

    \begin{tcolorbox}[breakable, size=fbox, boxrule=1pt, pad at break*=1mm,colback=cellbackground, colframe=cellborder]
\prompt{In}{incolor}{97}{\boxspacing}
\begin{Verbatim}[commandchars=\\\{\}]
\PY{n}{alphabets} \PY{o}{=} \PY{p}{[}\PY{l+s+s1}{\PYZsq{}}\PY{l+s+s1}{a}\PY{l+s+s1}{\PYZsq{}}\PY{p}{,} \PY{l+s+s1}{\PYZsq{}}\PY{l+s+s1}{b}\PY{l+s+s1}{\PYZsq{}}\PY{p}{,} \PY{l+s+s1}{\PYZsq{}}\PY{l+s+s1}{c}\PY{l+s+s1}{\PYZsq{}}\PY{p}{,} \PY{l+s+s1}{\PYZsq{}}\PY{l+s+s1}{d}\PY{l+s+s1}{\PYZsq{}}\PY{p}{,} \PY{l+s+s1}{\PYZsq{}}\PY{l+s+s1}{e}\PY{l+s+s1}{\PYZsq{}}\PY{p}{]}
\PY{n}{alphabet\PYZus{}dictionary} \PY{o}{=} \PY{p}{\PYZob{}}\PY{n}{\PYZus{}} \PY{o}{+} \PY{l+m+mi}{1}\PY{p}{:} \PY{n}{i} \PY{k}{for} \PY{n}{\PYZus{}}\PY{p}{,} \PY{n}{i} \PY{o+ow}{in} \PY{n+nb}{enumerate}\PY{p}{(}\PY{n}{alphabets}\PY{p}{)}\PY{p}{\PYZcb{}}
\PY{n}{alphabet\PYZus{}dictionary}
\end{Verbatim}
\end{tcolorbox}

            \begin{tcolorbox}[breakable, size=fbox, boxrule=.5pt, pad at break*=1mm, opacityfill=0]
\prompt{Out}{outcolor}{97}{\boxspacing}
\begin{Verbatim}[commandchars=\\\{\}]
\{1: 'a', 2: 'b', 3: 'c', 4: 'd', 5: 'e'\}
\end{Verbatim}
\end{tcolorbox}
        
    \hypertarget{accessing-elements-from-a-dictionary}{%
\subsubsection{Accessing elements from a
dictionary}\label{accessing-elements-from-a-dictionary}}

    \begin{tcolorbox}[breakable, size=fbox, boxrule=1pt, pad at break*=1mm,colback=cellbackground, colframe=cellborder]
\prompt{In}{incolor}{98}{\boxspacing}
\begin{Verbatim}[commandchars=\\\{\}]
\PY{c+c1}{\PYZsh{} usually}
\PY{n+nb}{print}\PY{p}{(}\PY{n}{alphabet\PYZus{}dictionary}\PY{p}{[}\PY{l+m+mi}{1}\PY{p}{]}\PY{p}{)}

\PY{c+c1}{\PYZsh{} or the get method}
\PY{n+nb}{print}\PY{p}{(}\PY{n}{alphabet\PYZus{}dictionary}\PY{o}{.}\PY{n}{get}\PY{p}{(}\PY{l+m+mi}{5}\PY{p}{)}\PY{p}{)}
\end{Verbatim}
\end{tcolorbox}

    \begin{Verbatim}[commandchars=\\\{\}]
a
e
    \end{Verbatim}

    \hypertarget{adding-items-to-a-dictionary}{%
\subsubsection{Adding items to a
dictionary}\label{adding-items-to-a-dictionary}}

    \begin{tcolorbox}[breakable, size=fbox, boxrule=1pt, pad at break*=1mm,colback=cellbackground, colframe=cellborder]
\prompt{In}{incolor}{99}{\boxspacing}
\begin{Verbatim}[commandchars=\\\{\}]
\PY{n}{alphabet\PYZus{}dictionary}\PY{p}{[}\PY{l+m+mi}{6}\PY{p}{]} \PY{o}{=} \PY{l+s+s1}{\PYZsq{}}\PY{l+s+s1}{f}\PY{l+s+s1}{\PYZsq{}}
\PY{n}{alphabet\PYZus{}dictionary}
\end{Verbatim}
\end{tcolorbox}

            \begin{tcolorbox}[breakable, size=fbox, boxrule=.5pt, pad at break*=1mm, opacityfill=0]
\prompt{Out}{outcolor}{99}{\boxspacing}
\begin{Verbatim}[commandchars=\\\{\}]
\{1: 'a', 2: 'b', 3: 'c', 4: 'd', 5: 'e', 6: 'f'\}
\end{Verbatim}
\end{tcolorbox}
        
    \hypertarget{some-functions-of-a-dictionary}{%
\subsubsection{Some functions of a
dictionary}\label{some-functions-of-a-dictionary}}

    \begin{tcolorbox}[breakable, size=fbox, boxrule=1pt, pad at break*=1mm,colback=cellbackground, colframe=cellborder]
\prompt{In}{incolor}{100}{\boxspacing}
\begin{Verbatim}[commandchars=\\\{\}]
\PY{n}{alphabet\PYZus{}dictionary}\PY{p}{[}\PY{l+m+mi}{6}\PY{p}{]} \PY{o}{=} \PY{l+s+s1}{\PYZsq{}}\PY{l+s+s1}{f}\PY{l+s+s1}{\PYZsq{}}

\PY{c+c1}{\PYZsh{} deleting an element}
\PY{n+nb}{print}\PY{p}{(}\PY{n}{alphabet\PYZus{}dictionary}\PY{o}{.}\PY{n}{pop}\PY{p}{(}\PY{l+m+mi}{6}\PY{p}{)}\PY{p}{)}
\PY{n+nb}{print}\PY{p}{(}\PY{n}{alphabet\PYZus{}dictionary}\PY{p}{)}

\PY{n+nb}{print}\PY{p}{(}\PY{n}{alphabet\PYZus{}dictionary}\PY{o}{.}\PY{n}{popitem}\PY{p}{(}\PY{p}{)}\PY{p}{)}
\PY{n+nb}{print}\PY{p}{(}\PY{n}{alphabet\PYZus{}dictionary}\PY{p}{)}

\PY{n}{keys} \PY{o}{=} \PY{n}{alphabet\PYZus{}dictionary}\PY{o}{.}\PY{n}{keys}\PY{p}{(}\PY{p}{)}
\PY{n+nb}{print}\PY{p}{(}\PY{n}{keys}\PY{p}{)}
\PY{n+nb}{print}\PY{p}{(}\PY{n+nb}{type}\PY{p}{(}\PY{n}{keys}\PY{p}{)}\PY{p}{)}

\PY{n}{values} \PY{o}{=} \PY{n}{alphabet\PYZus{}dictionary}\PY{o}{.}\PY{n}{values}\PY{p}{(}\PY{p}{)}
\PY{n+nb}{print}\PY{p}{(}\PY{n}{values}\PY{p}{)}
\PY{n+nb}{print}\PY{p}{(}\PY{n+nb}{type}\PY{p}{(}\PY{n}{values}\PY{p}{)}\PY{p}{)}

\PY{n}{items} \PY{o}{=} \PY{n}{alphabet\PYZus{}dictionary}\PY{o}{.}\PY{n}{items}\PY{p}{(}\PY{p}{)}
\PY{n+nb}{print}\PY{p}{(}\PY{n}{items}\PY{p}{)}
\PY{n+nb}{print}\PY{p}{(}\PY{n+nb}{type}\PY{p}{(}\PY{n}{items}\PY{p}{)}\PY{p}{)}
\end{Verbatim}
\end{tcolorbox}

    \begin{Verbatim}[commandchars=\\\{\}]
f
\{1: 'a', 2: 'b', 3: 'c', 4: 'd', 5: 'e'\}
(5, 'e')
\{1: 'a', 2: 'b', 3: 'c', 4: 'd'\}
dict\_keys([1, 2, 3, 4])
<class 'dict\_keys'>
dict\_values(['a', 'b', 'c', 'd'])
<class 'dict\_values'>
dict\_items([(1, 'a'), (2, 'b'), (3, 'c'), (4, 'd')])
<class 'dict\_items'>
    \end{Verbatim}

    \hypertarget{assignment-things}{%
\subsection{Assignment things}\label{assignment-things}}

    \begin{tcolorbox}[breakable, size=fbox, boxrule=1pt, pad at break*=1mm,colback=cellbackground, colframe=cellborder]
\prompt{In}{incolor}{126}{\boxspacing}
\begin{Verbatim}[commandchars=\\\{\}]
\PY{n}{my\PYZus{}dict} \PY{o}{=} \PY{p}{\PYZob{}}\PY{l+m+mi}{1}\PY{p}{:} \PY{l+s+s2}{\PYZdq{}}\PY{l+s+s2}{Ramesh}\PY{l+s+s2}{\PYZdq{}}\PY{p}{,} \PY{l+m+mi}{2}\PY{p}{:} \PY{l+s+s2}{\PYZdq{}}\PY{l+s+s2}{Suresh}\PY{l+s+s2}{\PYZdq{}}\PY{p}{,} \PY{l+m+mi}{3}\PY{p}{:} \PY{l+s+s2}{\PYZdq{}}\PY{l+s+s2}{Rajesh}\PY{l+s+s2}{\PYZdq{}}\PY{p}{,} \PY{l+m+mi}{4}\PY{p}{:} \PY{l+s+s2}{\PYZdq{}}\PY{l+s+s2}{Rakesh}\PY{l+s+s2}{\PYZdq{}}\PY{p}{,} \PY{l+m+mi}{5}\PY{p}{:} \PY{l+s+s2}{\PYZdq{}}\PY{l+s+s2}{Mahesh}\PY{l+s+s2}{\PYZdq{}}\PY{p}{,} \PY{l+m+mi}{6}\PY{p}{:} \PY{l+s+s2}{\PYZdq{}}\PY{l+s+s2}{Ganesh}\PY{l+s+s2}{\PYZdq{}}\PY{p}{\PYZcb{}}
\PY{k}{for} \PY{n}{key}\PY{p}{,} \PY{n}{val} \PY{o+ow}{in} \PY{n}{my\PYZus{}dict}\PY{o}{.}\PY{n}{items}\PY{p}{(}\PY{p}{)}\PY{p}{:} 
    \PY{n+nb}{print}\PY{p}{(}\PY{n}{key}\PY{p}{,} \PY{n}{val}\PY{p}{)}
\end{Verbatim}
\end{tcolorbox}

    \begin{Verbatim}[commandchars=\\\{\}]
1 Ramesh
2 Suresh
3 Rajesh
4 Rakesh
5 Mahesh
6 Ganesh
    \end{Verbatim}

    \begin{tcolorbox}[breakable, size=fbox, boxrule=1pt, pad at break*=1mm,colback=cellbackground, colframe=cellborder]
\prompt{In}{incolor}{127}{\boxspacing}
\begin{Verbatim}[commandchars=\\\{\}]
\PY{c+c1}{\PYZsh{} to remove items from a dictionary}
\PY{n}{my\PYZus{}dict}\PY{o}{.}\PY{n}{popitem}\PY{p}{(}\PY{p}{)}
\PY{n+nb}{print}\PY{p}{(}\PY{n}{my\PYZus{}dict}\PY{p}{)}
\PY{n}{my\PYZus{}dict}\PY{p}{[}\PY{l+m+mi}{7}\PY{p}{]} \PY{o}{=} \PY{l+s+s1}{\PYZsq{}}\PY{l+s+s1}{Paresh}\PY{l+s+s1}{\PYZsq{}}
\PY{n}{my\PYZus{}dict}
\end{Verbatim}
\end{tcolorbox}

    \begin{Verbatim}[commandchars=\\\{\}]
\{1: 'Ramesh', 2: 'Suresh', 3: 'Rajesh', 4: 'Rakesh', 5: 'Mahesh'\}
    \end{Verbatim}

            \begin{tcolorbox}[breakable, size=fbox, boxrule=.5pt, pad at break*=1mm, opacityfill=0]
\prompt{Out}{outcolor}{127}{\boxspacing}
\begin{Verbatim}[commandchars=\\\{\}]
\{1: 'Ramesh', 2: 'Suresh', 3: 'Rajesh', 4: 'Rakesh', 5: 'Mahesh', 7: 'Paresh'\}
\end{Verbatim}
\end{tcolorbox}
        
    \hypertarget{dict-from-list}{%
\subsection{2. dict from list}\label{dict-from-list}}

    \begin{tcolorbox}[breakable, size=fbox, boxrule=1pt, pad at break*=1mm,colback=cellbackground, colframe=cellborder]
\prompt{In}{incolor}{128}{\boxspacing}
\begin{Verbatim}[commandchars=\\\{\}]
\PY{n}{list1} \PY{o}{=} \PY{p}{[}\PY{l+s+s1}{\PYZsq{}}\PY{l+s+s1}{name}\PY{l+s+s1}{\PYZsq{}}\PY{p}{,} \PY{l+s+s1}{\PYZsq{}}\PY{l+s+s1}{panel}\PY{l+s+s1}{\PYZsq{}}\PY{p}{,} \PY{l+s+s1}{\PYZsq{}}\PY{l+s+s1}{rollno}\PY{l+s+s1}{\PYZsq{}}\PY{p}{]}
\PY{n}{list2} \PY{o}{=} \PY{p}{[}\PY{l+s+s1}{\PYZsq{}}\PY{l+s+s1}{Ramesh}\PY{l+s+s1}{\PYZsq{}}\PY{p}{,} \PY{l+s+s1}{\PYZsq{}}\PY{l+s+s1}{A}\PY{l+s+s1}{\PYZsq{}}\PY{p}{,} \PY{l+m+mi}{1}\PY{p}{]}
\PY{n}{my\PYZus{}dict} \PY{o}{=} \PY{p}{\PYZob{}}\PY{n}{i}\PY{p}{:} \PY{n}{j} \PY{k}{for} \PY{n}{i}\PY{p}{,} \PY{n}{j} \PY{o+ow}{in} \PY{n+nb}{zip}\PY{p}{(}\PY{n}{list1}\PY{p}{,} \PY{n}{list2}\PY{p}{)}\PY{p}{\PYZcb{}}
\PY{n}{my\PYZus{}dict}
\end{Verbatim}
\end{tcolorbox}

            \begin{tcolorbox}[breakable, size=fbox, boxrule=.5pt, pad at break*=1mm, opacityfill=0]
\prompt{Out}{outcolor}{128}{\boxspacing}
\begin{Verbatim}[commandchars=\\\{\}]
\{'name': 'Ramesh', 'panel': 'A', 'rollno': 1\}
\end{Verbatim}
\end{tcolorbox}
        
    \hypertarget{sort-elements}{%
\subsection{3. sort elements}\label{sort-elements}}

    \begin{tcolorbox}[breakable, size=fbox, boxrule=1pt, pad at break*=1mm,colback=cellbackground, colframe=cellborder]
\prompt{In}{incolor}{129}{\boxspacing}
\begin{Verbatim}[commandchars=\\\{\}]
\PY{n}{sorted\PYZus{}dictionary} \PY{o}{=} \PY{p}{\PYZob{}}\PY{n}{i}\PY{p}{:} \PY{n}{my\PYZus{}dict}\PY{o}{.}\PY{n}{get}\PY{p}{(}\PY{n}{i}\PY{p}{)} \PY{k}{for} \PY{n}{i} \PY{o+ow}{in} \PY{n+nb}{sorted}\PY{p}{(}\PY{n}{my\PYZus{}dict}\PY{p}{)}\PY{p}{\PYZcb{}}
\PY{n}{sorted\PYZus{}dictionary}
\end{Verbatim}
\end{tcolorbox}

            \begin{tcolorbox}[breakable, size=fbox, boxrule=.5pt, pad at break*=1mm, opacityfill=0]
\prompt{Out}{outcolor}{129}{\boxspacing}
\begin{Verbatim}[commandchars=\\\{\}]
\{'name': 'Ramesh', 'panel': 'A', 'rollno': 1\}
\end{Verbatim}
\end{tcolorbox}
        
    \hypertarget{make-1-list-of-keys-and-other-list-of-values}{%
\subsection{4. Make 1 list of keys and other list of
values}\label{make-1-list-of-keys-and-other-list-of-values}}

    \begin{tcolorbox}[breakable, size=fbox, boxrule=1pt, pad at break*=1mm,colback=cellbackground, colframe=cellborder]
\prompt{In}{incolor}{132}{\boxspacing}
\begin{Verbatim}[commandchars=\\\{\}]
\PY{n+nb}{print}\PY{p}{(}\PY{n}{my\PYZus{}dict}\PY{p}{)}
\PY{n}{list\PYZus{}keys} \PY{o}{=} \PY{n+nb}{list}\PY{p}{(}\PY{n}{my\PYZus{}dict}\PY{o}{.}\PY{n}{keys}\PY{p}{(}\PY{p}{)}\PY{p}{)}
\PY{n}{list\PYZus{}values} \PY{o}{=} \PY{n+nb}{list}\PY{p}{(}\PY{n}{my\PYZus{}dict}\PY{o}{.}\PY{n}{values}\PY{p}{(}\PY{p}{)}\PY{p}{)}

\PY{n+nb}{print}\PY{p}{(}\PY{n}{list\PYZus{}keys}\PY{p}{)}
\PY{n+nb}{print}\PY{p}{(}\PY{n}{list\PYZus{}values}\PY{p}{)}
\end{Verbatim}
\end{tcolorbox}

    \begin{Verbatim}[commandchars=\\\{\}]
\{'name': 'Ramesh', 'panel': 'A', 'rollno': 1\}
['name', 'panel', 'rollno']
['Ramesh', 'A', 1]
    \end{Verbatim}

    \hypertarget{find-the-mean-value-of-the-dictionary}{%
\subsection{5. Find the mean value of the
dictionary}\label{find-the-mean-value-of-the-dictionary}}

    \begin{tcolorbox}[breakable, size=fbox, boxrule=1pt, pad at break*=1mm,colback=cellbackground, colframe=cellborder]
\prompt{In}{incolor}{133}{\boxspacing}
\begin{Verbatim}[commandchars=\\\{\}]
\PY{n}{marks} \PY{o}{=} \PY{p}{\PYZob{}}
    \PY{l+s+s1}{\PYZsq{}}\PY{l+s+s1}{marks1}\PY{l+s+s1}{\PYZsq{}} \PY{p}{:} \PY{l+m+mi}{90}\PY{p}{,}
    \PY{l+s+s1}{\PYZsq{}}\PY{l+s+s1}{marks2}\PY{l+s+s1}{\PYZsq{}} \PY{p}{:} \PY{l+m+mi}{80}\PY{p}{,}
    \PY{l+s+s1}{\PYZsq{}}\PY{l+s+s1}{marks3}\PY{l+s+s1}{\PYZsq{}} \PY{p}{:} \PY{l+m+mi}{80}\PY{p}{,}
    \PY{l+s+s1}{\PYZsq{}}\PY{l+s+s1}{marks4}\PY{l+s+s1}{\PYZsq{}} \PY{p}{:} \PY{l+m+mi}{80}\PY{p}{,}
    \PY{l+s+s1}{\PYZsq{}}\PY{l+s+s1}{marks5}\PY{l+s+s1}{\PYZsq{}} \PY{p}{:} \PY{l+m+mi}{80}\PY{p}{,}
\PY{p}{\PYZcb{}}
\PY{n}{mean\PYZus{}value} \PY{o}{=} \PY{n+nb}{sum}\PY{p}{(}\PY{n}{marks}\PY{o}{.}\PY{n}{values}\PY{p}{(}\PY{p}{)}\PY{p}{)} \PY{o}{/} \PY{n+nb}{len}\PY{p}{(}\PY{n}{marks}\PY{p}{)}
\PY{n+nb}{print}\PY{p}{(}\PY{n}{mean\PYZus{}value}\PY{p}{)}
\end{Verbatim}
\end{tcolorbox}

    \begin{Verbatim}[commandchars=\\\{\}]
82.0
    \end{Verbatim}

    \hypertarget{perform-the-following-operations-on-the-dictionary}{%
\subsection{6. Perform the following operations on the
dictionary}\label{perform-the-following-operations-on-the-dictionary}}

    \begin{tcolorbox}[breakable, size=fbox, boxrule=1pt, pad at break*=1mm,colback=cellbackground, colframe=cellborder]
\prompt{In}{incolor}{135}{\boxspacing}
\begin{Verbatim}[commandchars=\\\{\}]
\PY{n}{my\PYZus{}dict} \PY{o}{=} \PY{p}{\PYZob{}}\PY{l+s+s1}{\PYZsq{}}\PY{l+s+s1}{name}\PY{l+s+s1}{\PYZsq{}}\PY{p}{:} \PY{p}{[}\PY{l+s+s1}{\PYZsq{}}\PY{l+s+s1}{Yash}\PY{l+s+s1}{\PYZsq{}}\PY{p}{,} \PY{l+s+s1}{\PYZsq{}}\PY{l+s+s1}{Neal}\PY{l+s+s1}{\PYZsq{}}\PY{p}{,} \PY{l+s+s1}{\PYZsq{}}\PY{l+s+s1}{Dev}\PY{l+s+s1}{\PYZsq{}}\PY{p}{]}\PY{p}{,} \PY{l+s+s1}{\PYZsq{}}\PY{l+s+s1}{Roll}\PY{l+s+s1}{\PYZsq{}}\PY{p}{:} \PY{p}{[}\PY{l+m+mi}{1}\PY{p}{,} \PY{l+m+mi}{12}\PY{p}{,} \PY{l+m+mi}{3}\PY{p}{]}\PY{p}{,} \PY{l+s+s1}{\PYZsq{}}\PY{l+s+s1}{Marks}\PY{l+s+s1}{\PYZsq{}}\PY{p}{:} \PY{p}{[}\PY{l+m+mi}{90}\PY{p}{,} \PY{l+m+mi}{80}\PY{p}{,} \PY{l+m+mi}{70}\PY{p}{]}\PY{p}{\PYZcb{}}
\PY{n+nb}{print}\PY{p}{(}\PY{l+s+s2}{\PYZdq{}}\PY{l+s+s2}{Name is: }\PY{l+s+s2}{\PYZdq{}}\PY{p}{,} \PY{n}{my\PYZus{}dict}\PY{p}{[}\PY{l+s+s1}{\PYZsq{}}\PY{l+s+s1}{name}\PY{l+s+s1}{\PYZsq{}}\PY{p}{]}\PY{p}{[}\PY{l+m+mi}{2}\PY{p}{]}\PY{p}{)}
\PY{n+nb}{print}\PY{p}{(}\PY{l+s+s2}{\PYZdq{}}\PY{l+s+s2}{Roll is: }\PY{l+s+s2}{\PYZdq{}}\PY{p}{,} \PY{n}{my\PYZus{}dict}\PY{p}{[}\PY{l+s+s1}{\PYZsq{}}\PY{l+s+s1}{Roll}\PY{l+s+s1}{\PYZsq{}}\PY{p}{]}\PY{p}{[}\PY{l+m+mi}{1}\PY{p}{]}\PY{p}{)}
\PY{n+nb}{print}\PY{p}{(}\PY{l+s+s2}{\PYZdq{}}\PY{l+s+s2}{Highest Marks are: }\PY{l+s+s2}{\PYZdq{}}\PY{p}{,} \PY{n+nb}{max}\PY{p}{(}\PY{n}{my\PYZus{}dict}\PY{p}{[}\PY{l+s+s1}{\PYZsq{}}\PY{l+s+s1}{Marks}\PY{l+s+s1}{\PYZsq{}}\PY{p}{]}\PY{p}{)}\PY{p}{)}
\end{Verbatim}
\end{tcolorbox}

    \begin{Verbatim}[commandchars=\\\{\}]
Name is:  Dev
Roll is:  12
Highest Marks are:  90
    \end{Verbatim}


\section{Conclusion}
The List data structure in python was studied in detail. The different operations that can be performed on a list were also studied. The code was written and tested to check the output. The differences between lists and tuples were also studied.

\clearpage

\section{FAQ}

\begin{enumerate}
	\item \textbf{What will be the output of the following code snippet?}\\
	      \begin{verbatim}
		a=[1,2,3,4,5,6,7,8,9]
		print(a[::2])

		> [1, 3, 5, 7, 9]
	\end{verbatim}
	\item \textbf{What will be the output of the following code snippet?}\\
	      \begin{verbatim}
		l = [1, 2, 3]
		init_tuple = ('Python',) * (l.__len__() - l[::-1][0])
		print(init_tuple)

		> ()
	\end{verbatim}

	\item \textbf{State the difference between list and dictionary.}\\
	      \begin{enumerate}
		      \item List is mutable whereas dictionary is immutable.
		      \item List is ordered whereas dictionary is unordered.
		      \item List is indexed whereas dictionary is not indexed.
		      \item List is iterable whereas dictionary is not iterable.
		      \item List is a data structure that stores a sequence of values whereas dictionary is a data structure that stores a sequence of key-value pairs.
	      \end{enumerate}

\end{enumerate}


\end{document}