% This is a Basic Assignment Paper but with like Code and stuff allowed in it, there is also url, hyperlinks from contents included. 

\documentclass[11pt]{article}

% Preamble

\usepackage[margin=1in]{geometry}
\usepackage{amsfonts, amsmath, amssymb}
\usepackage{fancyhdr, float, graphicx}
\usepackage[utf8]{inputenc} % Required for inputting international characters
\usepackage[T1]{fontenc} % Output font encoding for international characters
\usepackage{fouriernc} % Use the New Century Schoolbook font
\usepackage[nottoc, notlot, notlof]{tocbibind}
\usepackage{listings}
\usepackage{xcolor}
\usepackage{blindtext}
\usepackage{hyperref}
\usepackage{booktabs}
\hypersetup{
    colorlinks=true,
    linkcolor=black,
    filecolor=magenta,      
    urlcolor=cyan,
    pdfpagemode=FullScreen,
    }

\definecolor{codegreen}{rgb}{0,0.6,0}
\definecolor{codegray}{rgb}{0.5,0.5,0.5}
\definecolor{codepurple}{rgb}{0.58,0,0.82}
\definecolor{backcolour}{rgb}{0.95,0.95,0.92}

\lstdefinestyle{mystyle}{
    backgroundcolor=\color{backcolour},   
    commentstyle=\color{codegreen},
    keywordstyle=\color{magenta},
    numberstyle=\tiny\color{codegray},
    stringstyle=\color{codepurple},
    basicstyle=\ttfamily\footnotesize,
    breakatwhitespace=false,         
    breaklines=true,                 
    captionpos=b,                    
    keepspaces=true,                 
    numbers=left,                    
    numbersep=5pt,                  
    showspaces=false,                
    showstringspaces=false,
    showtabs=false,                  
    tabsize=2
}

\lstset{style=mystyle}

% Header and Footer
\pagestyle{fancy}
\fancyhead{}
\fancyfoot{}
\fancyhead[L]{\textit{\Large{Advanced Data Structures - Assignment 7}}}
%\fancyhead[R]{\textit{something}}
\fancyfoot[C]{\thepage}
\renewcommand{\footrulewidth}{1pt}



% Other Doc Editing
% \parindent 0ex
%\renewcommand{\baselinestretch}{1.5}

\begin{document}

\begin{titlepage}
    \centering

    %---------------------------NAMES-------------------------------

    \huge\textsc{
        MIT World Peace University
    }\\

    \vspace{0.75\baselineskip} % space after Uni Name

    \LARGE{
        Advanced Data Structures\\
        Second Year B. Tech, Semester 4
    }

    \vfill % space after Sub Name

    %--------------------------TITLE-------------------------------

    \rule{\textwidth}{1.6pt}\vspace*{-\baselineskip}\vspace*{2pt}
    \rule{\textwidth}{0.6pt}
    \vspace{0.75\baselineskip} % Whitespace above the title



    \huge{\textsc{
            Minimum Cost Spanning Tree using
            \textit{Prim's Algorithm}
        }} \\



    \vspace{0.5\baselineskip} % Whitespace below the title
    \rule{\textwidth}{0.6pt}\vspace*{-\baselineskip}\vspace*{2.8pt}
    \rule{\textwidth}{1.6pt}

    \vspace{1\baselineskip} % Whitespace after the title block

    %--------------------------SUBTITLE --------------------------	

    \LARGE\textsc{
        Assignment No. 6
    } % Subtitle or further description
    \vfill

    %--------------------------AUTHOR-------------------------------

    Prepared By
    \vspace{0.5\baselineskip} % Whitespace before the editors

    \Large{
        Krishnaraj Thadesar \\
        Cyber Security and Forensics\\
        Batch A1, PA 20
    }


    \vspace{0.5\baselineskip} % Whitespace below the editor list
    \today

\end{titlepage}

\tableofcontents
\thispagestyle{empty}
\clearpage

\setcounter{page}{1}

\section{Objectives}
\begin{enumerate}
    \item To study the concept of heap
    \item To study different types of heap and their algorithms
\end{enumerate}

\section{Problem Statement}
\textit{Read the marks obtained by students of second year in an online examination of
    particular subject. Find out maximum and minimum marks obtained in that subject. Use
    heap data structure and Heap sort.}
\section{Theory}

% ● What is Heap?
% ● Write different types of Heaps
% ● Construction of heap and Data Structure used for creation.

\section{Platform}
\textbf{\textbf{Operating System}}: Arch Linux x86-64 \\
\textbf{\textbf{IDEs or Text Editors Used}}: Visual Studio Code\\
\textbf{\textbf{Compilers} }: g++ and gcc on linux for C++\\

\section{Test Conditions}
\begin{enumerate}
    \item Input min 10 elements.
    \item Display Max and Min Heap
    \item Find Maximum and Minimum marks obtained in a particular subject.
\end{enumerate}

\section{Input and Output}
\begin{enumerate}
    \item The minimum cost of the spanning tree.
\end{enumerate}

\section{Pseudo Code}
\subsubsection*{Pseudo Code for Creation of Min and Max Heap}
\begin{lstlisting}[language=c++]

\end{lstlisting}
\subsubsection*{Pseudo Code for Heapify function}
\begin{lstlisting}[language=c++]

\end{lstlisting}


\section{Time Complexity}

\subsection{Min and Max Heap Creation}
\begin{itemize}
    \item \textbf{Time Complexity:} \[O(V^2)\]
    \item \textbf{Space Complexity:} \[O(V^2)\]
\end{itemize}

\subsection{Min or Max Heap Traversal}

\begin{itemize}
    \item \textbf{Time Complexity:} \[O(V^2)\]
    \item \textbf{Space Complexity:} \[O(V^2)\]
\end{itemize}

\section{Code}

\subsection{Program}
\lstinputlisting[language=C++]{../Programs/Assignment_7.cpp}

\lstinputlisting[]{../Programs/Assignment_7_output.txt}

\section{Conclusion}
Thus, we have understood the importance and use of Heaps as a Data structure, and how they are better and more efficient than Binary Search Trees. We have also understood the working of Heap Sort and how it is implemented.

\clearpage

\section{FAQ}
\begin{enumerate}
    \item \textbf{Discuss with suitable example for heap sort?}\\

    \item \textbf{Compute the time complexity of heap sort?}\\

\end{enumerate}

\end{document}